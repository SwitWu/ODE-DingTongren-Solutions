\chapter{线性微分方程组}



\section{一般理论}



\subsection{证明与总结}



P159证明: $H(C_1\bm{y}_1^0+C_2\bm{y}_2^0)=C_1H(\bm{y}_1^0)+C_2H(\bm{y}_2^0)$.
\begin{proof} 首先显然$H(C_1\bm{y}_1^0+C_2\bm{y}_2^0)\in\mathcal{S}$, 
又由引理6.1知$C_1H(\bm{y}_1^0)+C_2H(\bm{y}_2^0)\in\mathcal{S}$.然后又因为
\[\begin{split}\left(H(C_1\bm{y}_1^0+C_2\bm{y}_2^0)\right)(x_0)&=C_1\bm{y}_1^0+C_2\bm{y}_2^0\\
\left(C_1H(\bm{y}_1^0)+C_2H(\bm{y}_2^0)\right)(x_0)&=C_1\bm{y}_1^0+C_2\bm{y}_2^0
\end{split}\]
由解的唯一性知$H(C_1\bm{y}_1^0+C_2\bm{y}_2^0)=C_1H(\bm{y}_1^0)+C_2H(\bm{y}_2^0)$.

注意解矩阵的行列式就是其对应的解组的Wronsky行列式.
\end{proof}

推论6.2的证明:

\begin{proof}
(1) 记 $\bm{C}=(\bm{c}_1,\cdots,\bm{c}_n)$,
则 $\bm{\varPhi}(x)\bm{C}=(\bm{\varPhi}(x)\bm{c}_1,\cdots,\bm{\varPhi}(x)\bm{c}_n)$,
由 (6.15) 式知 $\forall 1\leq i\leq n$, $\bm{\varPhi}(x)\bm{c}_i$ 是方程(6.2)的解,
也即 $\{\bm{\varPhi}(x)\bm{c}_i|1\leq i\leq n\}$ 为(6.2)的解组.
记 $\bmitPhi(x)$ 对应的基本解组的 Wronsky 行列式为 $W(x)$,
则 $\bmitPhi(x)\bm{C}$ 对应的解组的 Wronsky 行列式为:
\[W_1(x) = |\bmitPhi(x)\bm{c}_1\cdots\bmitPhi(x)\bm{c}_n|
  = |\bmitPhi(x)|\cdot|\bm{C}|=W(x)|\bm{C}|\neq 0.\]
故 $\bmitPhi(x)\bm{C}$也是基解矩阵.

(2) $\bmitPhi(x)$ 和 $\bmitPsi(x)$ 都是基解矩阵, 设
\[\bmitPhi(x)=(\bm{y}_1(x),\cdots,\bm{y}_n(x)),\]
\[\bmitPsi(x)=(\bm{y}_1^*(x),\cdots,\bm{y}_n^*(x)).\]
因为 $\bmitPhi(x)$ 是基解矩阵, 所以 $\bm{y}_1(x),\cdots,\bm{y}_n(x)$ 是 (6.2) 的基本解组, 
故存在 $\{c_{ij}\mid 1\leq i,j\leq n\}$ 使得
\[\bm{y}_i^*(x)=\sum_{j=1}^nc_{ji}\bm{y}_j(x),1\leq i\leq n.\]
即
\[(\bm{y}_1^*(x),\cdots,\bm{y}_n^*(x)) = (\bm{y}_1(x),\cdots,\bm{y}_n(x))
\begin{pmatrix}
  c_{11}&c_{12}&\cdots&c_{1n}\\
  c_{21}&c_{22}&\cdots&c_{2n}\\
  \vdots&\vdots&&\vdots\\
  c_{n1}&c_{n2}&\cdots&c_{nn}
\end{pmatrix}\]
也即
\[\bmitPsi(x)=\bmitPhi(x)\bm{C},\]
其中 $\bm{C}=(c_{ij})_{n\times n}$, 在上述等式两边同时取行列式得$|\bmitPsi|=|\bmitPhi|\cdot|\bm{C}|$, 
由 $|\bmitPsi|\neq 0$, $|\bmitPhi|\neq 0$ 知 $|\bm{C}|\neq 0$.
\end{proof}



\subsection{习题}



\begin{exercise}
  求出齐次线性微分方程组
  \[\frac{\diff\bm{y}}{\diff x}=\bm{A}(t)\bm{y}\]
  的通解, 其中 $\bm{A}(t)$ 分别为:
  \begin{enumerate}[(1)]
  \item $\displaystyle\bm{A}(t)=\begin{pmatrix}\frac{1}{t}&0\\0&\frac{1}{t}\end{pmatrix},t\neq0$;
  \item $\displaystyle\bm{A}(t)=\begin{pmatrix}1&1\\0&1\end{pmatrix}$;
  \item $\displaystyle\bm{A}(t)=\begin{pmatrix}0&1\\-1&0\end{pmatrix}$;
  \item $\displaystyle\bm{A}(t)=\begin{pmatrix}0&0&1\\0&1&0\\1&0&0\end{pmatrix}$.
  \end{enumerate}
\end{exercise}

\begin{solution}
  (1) $\displaystyle\frac{\diff}{\diff t}\begin{pmatrix}y_1\\y_2\end{pmatrix}=\begin{pmatrix}\frac{1}{t}&0\\0&\frac{1}{t}\end{pmatrix}\begin{pmatrix}y_1\\y_2\end{pmatrix}$, 
  分量形式为 $\displaystyle\frac{\diff y_1}{\diff t}=\frac{y_1}{t},\frac{\diff y_2}{\diff t}=\frac{y_2}{t}$, 
  解得 $y_1=kt,y_2=kt$, 不妨取基解矩阵为 $\begin{pmatrix}0&t\\t&0\end{pmatrix}$, 则通解为
  \[\bm{y}(t)=C_1\begin{pmatrix}0\\t\end{pmatrix}+C_2\begin{pmatrix}t\\0\end{pmatrix}.\]

  (2) $\displaystyle\frac{\diff}{\diff t}\begin{pmatrix}y_1\\y_2\end{pmatrix}=\begin{pmatrix}1&1\\0&1\end{pmatrix}\begin{pmatrix}y_1\\y_2\end{pmatrix}$, 
  分量形式为$\displaystyle\frac{\diff y_1}{\diff t}=y_1+y_2,\frac{\diff y_2}{\diff t}=y_2$, 
  由$\displaystyle\frac{\diff y_2}{\diff t}=y_2$解得$y_2=C\e^t$, 先取$y_2=0$得$y_1=C\e^t$, 
  再取$y_2=\e^t$得$y_1=C\e^t+t\e^t$, 不妨取基解矩阵为$\begin{pmatrix}\e^t&t\e^t\\0&\e^t\end{pmatrix}$, 则通解为
  \[\bm{y}(t)=C_1\begin{pmatrix}\e^t\\0\end{pmatrix}+C_2\begin{pmatrix}t\e^t\\\e^t\end{pmatrix}\]

  (3) $\displaystyle\frac{\diff}{\diff t}\begin{pmatrix}y_1\\y_2\end{pmatrix}=\begin{pmatrix}0&1\\-1&0\end{pmatrix}\begin{pmatrix}y_1\\y_2\end{pmatrix}$, 
  分量形式为 $\displaystyle\frac{\diff y_1}{\diff t}=y_2,\frac{\diff y_2}{\diff t}=-y_1$, 
  故 $\displaystyle\frac{\diff^2y_1}{\diff t}=\frac{\diff y_2}{\diff t}=-y_1\Rightarrow y_1=C_1\sin t+C_2\cos t,y_2=C_1\cos t-C_2\sin t$, 
  不妨取基解矩阵为 $\begin{pmatrix}\sin t&\cos t\\\cos t&-\sin t\end{pmatrix}$, 则通解为
  \[\bm{y}(t)=C_1\begin{pmatrix}\sin t\\\cos t\end{pmatrix}+C_2\begin{pmatrix}\cos t\\-\sin t\end{pmatrix}.\]

  (4) $\displaystyle\frac{\diff}{\diff t}\begin{pmatrix}y_1\\y_2\\y_3\end{pmatrix}=\begin{pmatrix}0&0&1\\0&1&0\\1&0&0\end{pmatrix}\begin{pmatrix}y_1\\y_2\\y_3\end{pmatrix}$, 
  分量形式为 $\displaystyle\frac{\diff y_1}{\diff t}=y_3,\frac{\diff y_2}{\diff t}=y_2,\frac{\diff y_3}{\diff t}=y_1$, 
  由 $\displaystyle\frac{\diff y_2}{\diff t}=y_2$
  得 $y_2=C\e^t$, 又$\displaystyle\frac{\diff^2y_1}{\diff t^2}=\frac{\diff y_3}{\diff t}=y_1\Rightarrow y_1=C_1\e^t+C_2\e^{-t}$, 
  取 $y_1=\e^t$, 得 $y_3=\e^t$, 取 $y_1=\e^{-t}$, 得 $y_3=-\e^{-t}$, 
  不妨取基解矩阵为 $\begin{pmatrix}\e^t&0&\e^{-t}\\0&\e^t&0\\\e^t&0&-\e^{-t}\end{pmatrix}$, 则通解为
  \[\bm{y}(t)=C_1\begin{pmatrix}\e^t\\0\\\e^t\end{pmatrix}+C_2\begin{pmatrix}0\\\e^t\\0\end{pmatrix}+C_3\begin{pmatrix}\e^{-t}\\0\\-\e^{-t}\end{pmatrix}.\qedhere\]
\end{solution}



\begin{exercise}
  求解非齐次线性微分方程组的初值问题:

  (1) $\begin{cases}
    \frac{\diff x}{\diff t} = 1-\frac{2}{t}x,\quad\frac{\diff y}{\diff t}=x+y-1+\frac{2}{t}x\quad(t>0),\\
    x(1) = \frac{1}{3},\quad y(1)=-\frac{1}{3};\end{cases}$

  (2) $\begin{cases}
  \frac{\diff x}{\diff t}=\frac{2t}{1+t^2}x,\quad\frac{\diff y}{\diff t}=-\frac{1}{t}y+x+t\quad(t>0),\\
  x(1)=0,\quad y(1)=\frac{4}{3}.
  \end{cases}$.
\end{exercise}

\begin{solution}
  (1) 由 $\frac{\diff x}{\diff t}=1-\frac{2}{t}x$ 得 
  $x=\e^{-\int\frac{2}{t}\diff t}\left(C+\int\e^{\int\frac{2}{t}\diff t}\diff t\right)=\frac{t}{3}+\frac{C}{t^2}$, 
  代入初值条件 $x(1)=\frac{1}{3}$ 得 $x(t)=\frac{t}{3}$. 
  又 $\frac{\diff}{\diff t}(x+y)=x+y$ 且 $(x+y)(1)=0$, 
  故 $x+y\equiv 0$, 故 $y(t)=-\frac{t}{3}$.

  (2) 由 $\frac{\diff x}{\diff t}=\frac{2t}{1+t^2}x$ 得 $x(t)=C(1+t^2)$, 
  代入初值条件 $x(1)=0$ 得 $x(t)\equiv 0$, 
  故 $\frac{\diff y}{\diff t}=-\frac{y}{t}+t$, 
  解得 $y(t)=\frac{C}{t}+\frac{t^2}{3}$, 
  代入初值条件 $y(1)=\frac{4}{3}$ 得 $y(t)=\frac{1}{t}+\frac{t^2}{3}$.
\end{solution}



\begin{exercise}
  试证向量函数组
  \[\begin{pmatrix}1\\0\\0\end{pmatrix},\begin{pmatrix}x\\0\\0\end{pmatrix},\begin{pmatrix}x^2\\0\\0\end{pmatrix}\]
  在任意区间 $a<x<b$ 上线性无关. (显然, 它们的朗斯基行列式 $W(x)\equiv 0$. 对照定理 6.2 可知, 
  上述三个线性无关的向量函数不可能同时满足任意一个三阶的齐次线性微分方程组.)
\end{exercise}

\begin{proof}
  设 $k_1,k_2,k_3$ 满足
  \[k_1\begin{pmatrix}1\\0\\0\end{pmatrix}+k_2\begin{pmatrix}x\\0\\0\end{pmatrix}+k_3\begin{pmatrix}x^2\\0\\0\end{pmatrix}=\bm{0},\quad a<x<b.\]
  即 $k_1+k_2x+k_3x^2=0$, $a<x<b$, 显然必有 $k_1=k_2=k_3=0$, 故该向量函数组线性无关.
\end{proof}



\begin{exercise}
  试证基解矩阵完全决定齐次线性微分方程组, 即如果方程组
  \[\frac{\diff\bm{y}}{\diff x}=\bm{A}(x)\bm{y}\quad\mbox{与}\quad\frac{\diff\bm{y}}{\diff x}=\bm{B}(x)\bm{y}\]
  有一个相同的基解矩阵, 则 $\bm{A}(x)\equiv\bm{B}(x)$.
\end{exercise}

\begin{proof} 
  设相同的基解矩阵为 $\bmitPhi(x)$, 则
  \[\frac{\diff}{\diff x}\bmitPhi(x)=\bm{A}(x)\bmitPhi(x)=\bm{B}(x)\bmitPhi(x).\]
  由于 $\bmitPhi(x)$ 可逆, 故 $\bm{A}(x)\equiv\bm{B}(x)$.
\end{proof}



\begin{exercise}[5]
  设 $\bmitPhi(x)$ 是齐次线性微分方程组 (6.2) 的一个基解矩阵, 
  并且 $n$ 维向量函数 $\bm{f}(x,\bm{y})$ 在区域 $E(a<x<b,|y|<\infty)$ 上连续. 则求解初值问题
  \[\begin{cases}
  \frac{\diff\bm{y}}{\diff x}=\bm{A}(x)\bm{y}+\bm{f}(x,\bm{y}),\\\bm{y}(x_0)=\bm{y}_0
  \end{cases}\]
  等价于求解 (向量形式的) 积分方程
  \[\bm{y}(x)=\bmitPhi(x)\bmitPhi^{-1}(x_0)\bm{y}_0+\int_{x_0}^x\bmitPhi(x)\bmitPhi^{-1}(s)\bm{f}(s,\bm{y}(s))\diff s,\]
  其中 $x_0\in(a,b)$.
\end{exercise}

\begin{proof}
  由 $\bmitPhi^{-1}(x)\bmitPhi(x)=\bm{I}$, 求导得
  \[\bmitPhi^{-1}(x)\frac{\diff\bmitPhi(x)}{\diff x}
    + \frac{\diff\bmitPhi^{-1}(x)}{\diff x}\bmitPhi(x)=\bm{0}.\]
  又
  \[\frac{\diff\bmitPhi(x)}{\diff x}=\bm{A}(x)\bmitPhi(x).\]
  故
  \[\bmitPhi^{-1}(x)\bm{A}(x)+\frac{\diff\bmitPhi^{-1}(x)}{\diff x}=\bm{0}.\]
  设 $\bm{y}(x)$ 是初值问题的解, 则
  \begin{align*}
  & \frac{\diff\bm{y}(x)}{\diff x}=\bm{A}(x)\bm{y}(x)+\bm{f}(x,\bm{y}(x)),\quad\bm{y}(x_0)=\bm{y}_0 \\
  \Longleftrightarrow{} & \bmitPhi^{-1}(x)\frac{\diff\bm{y}(x)}{\diff x}=\bmitPhi^{-1}(x)\bm{A}(x)\bm{y}(x)+\bmitPhi^{-1}(x)\bm{f}(x,\bm{y}(x)),\quad\bm{y}(x_0)=\bm{y}_0 \\
  \Longleftrightarrow{} & \bmitPhi^{-1}(x)\frac{\diff\bm{y(x)}}{\diff x}+\frac{\diff\bmitPhi^{-1}(x)}{\diff x}\bm{y}(x)=\bmitPhi^{-1}(x)\bm{f}(x,\bm{y}(x)),\quad\bm{y}(x_0)=\bm{y}_0 \\
  \Longleftrightarrow{} & \frac{\diff}{\diff x}\left(\bmitPhi^{-1}(x)\bm{y}(x)\right)=\bmitPhi^{-1}(x)\bm{f}(x,\bm{y}(x)),\quad\bm{y}(x_0)=\bm{y}_0 \\
  \Longleftrightarrow{} & \bmitPhi^{-1}(x)\bm{y}(x)=\bmitPhi^{-1}(x_0)\bm{y}_0+\int_{x_0}^x\bmitPhi^{-1}(s)\bm{f}(s,\bm{y}(s))\diff s \\
  \Longleftrightarrow{} & \bm{y}(x)=\bmitPhi(x)\bmitPhi^{-1}(x_0)\bm{y}_0+\int_{x_0}^x\bmitPhi(x)\bmitPhi^{-1}(s)\bm{f}(s,\bm{y}(s))\diff s.
  \end{align*}
  故 $\bm{y}(x)$ 是初值问题的解等价于 $\bm{y}(x)$ 是积分方程的解.
\end{proof}



\begin{exercise}
  设当 $a<x<b$ 时, 非齐次线性微分方程组 (6.1) 中的 $\bm{f}(x)$ 不恒为零. 
  证明 (6.1) 有且至多有 $n+1$ 个线性无关解.
\end{exercise}

\begin{proof}
  (i) 设 $\bm{\phi}_1(x),\ldots,\bm{\phi}_n(x)$ 为相应的齐次线性微分方程组的一个基本解组, 
  $\bm{\phi}_*(x)$ 为 (6.1) 的一个特解, 
  则 $\bm{y}_0(x)=\bm{\phi}_*(x),\bm{y}_1(x)=\bm{\phi}_1(x)+\bm{\phi}_*(x),\cdots,\bm{y}_n(x)=\bm{\phi}_n(x)+\bm{\phi}_*(x)$
  是 (6.1) 的 $n+1$ 个解, 下证这 $n+1$ 个解线性无关, 设
  \begin{equation}
    \sum_{i=0}^nk_i\bm{y}_i(x)=\sum_{i=1}^nk_i\bm{\phi}_i(x)+\left(\sum_{i=0}^nk_i\right)\bm{\phi}_*(x)
      = \bm{0}.\tag{$\star$}
  \end{equation}
  对上式求导得
  \begin{align*}
    & \sum_{i=1}^nk_1\bm{A}(x)\bm{\phi}_i(x)+\left(\sum_{i=0}^nk_i\right)\left(\bm{A}(x)\bm{\phi}_*(x)+\bm{f}(x)\right)\\
    ={} & \bm{A}(x)\left(\sum_{i=1}^nk_i\bm{\phi}_i(x)+\left(\sum_{i=0}^nk_i\right)\bm{\phi}_*(x)\right)+\left(\sum_{i=0}^nk_i\right)\bm{f}(x)\\
    ={} & \left(\sum_{i=0}^nk_i\right)\bm{f}(x)=\bm{0}\Rightarrow\sum_{i=0}^nk_i=0.
  \end{align*}
  代回 $(\star)$ 式得 $\sum_{i=1}^nk_i\bm{\phi}_i(x)=0$, 
  故 $k_i=0$ $(i=0,1,\cdots,n)$, 从而这 $n+1$ 个解线性无关.

  (ii) 设 $\bm{\phi}(x)$ 是 (6.1) 的任意一个解, 
  则 $\bm{\phi}(x)-\bm{y}_0(x),\cdots,\bm{\phi}(x)-\bm{y}_n(x)$
  是相应的齐次线性微分方程组的 $n+1$ 个解, 故其必线性相关, 即存在不全为零的 $k_0,k_1,\cdots,k_n$ 使得
  \[\sum_{i=0}^nk_i(\bm{\phi}(x)-\bm{y}_i(x))
    = \left(\sum_{i=0}^nk_i\right)\bm{\phi}(x)-\sum_{i=0}^nk_i\bm{y}_i(x)=0.\]
  显然 $\sum_{i=0}^nk_i\neq 0$, 否则与 $\bm{y}_0(x),\cdots,\bm{y}_n(x)$ 的线性无关性矛盾, 故
  \[\bm{\phi}(x)=\sum_{j=0}^n\frac{k_j}{\sum_{i=0}^nk_i}\bm{y}_j(x).\]
  因此线性无关解个数不超过 $n+1$.
\end{proof}



\section{常系数线性微分方程组}



\subsection{证明与总结}



\begin{conclusion}[教材 Page 170]
  令 $\mathcal{M}$ 为全体 $n$ 阶实矩阵组成的集合,
  在 $\mathcal{M}$ 中定义范数为: 对于 $\bm{A}\in\mathcal{M}$, 令
  \[\|\bm{A}\| = \sum_{i,j=1}^n |a_{ij}|.\]
  则 $(\mathcal{M},\|\cdot\|)$ 是 Banach 空间.
\end{conclusion}

\begin{proof}
  任取 $\mathcal{M}$ 中的 Cauchy 序列 $(\bm{A}_n)_{n\geq 1}$,
  即 $\forall\epsilon>0,\exists N>0,\forall m,n>N$, 有
  \[\|\bm{A}_m-\bm{A}_n\| = \sum_{i,j=1}^n \bigl|a_{ij}^{(m)} - a_{ij}^{(n)}\bigr|<\epsilon.\]
  故对任意给定的 $i,j$, 序列 $\bigl(a_{ij}^{(n)}\bigr)_{n\geq 1}$
  是 $\mathbb{R}$ 中的 Cauchy 序列, 必收敛, 记之为 $a_{ij}^{(n)}\to a_{ij}$ $(n\to\infty)$,
  则 $\bm{A}_n\to \bm{A}=(a_{ij})_{n\times n}$, 因此 $(\mathcal{M},\|\cdot\|)$ 是 Banach 空间.
\end{proof}



\begin{proposition}[教材 Page 171]
  矩阵 $\bm{A}$ 的幂级数
  \[\bm{E} + \bm{A} + \frac{1}{2!}\bm{A}^2 + \cdots +
    \frac{1}{k!}\bm{A}^k + \cdots\]
  是绝对收敛的.
\end{proposition}

\begin{proof}
  因为
  \begin{align*}
  & \|\bm{E}\|+\|\bm{A}\|+\|\frac{1}{2!}\bm{A}^2\|+\cdots+\|\frac{1}{k!}\bm{A}^k\|+\cdots\\
  \leq{} & \|\bm{E}\|+\|\bm{A}\|+\frac{1}{2!}\|\bm{A}\|^2+\cdots+\frac{1}{k!}\|\bm{A}\|^k+\cdots\\
  ={} & n+\e^{\|\bm{A}\|}-1<\infty,
  \end{align*}
  故矩阵 $\bm{A}$ 的幂级数 
  $\bm{E}+\bm{A}+\frac{1}{2!}\bm{A}^2+\cdots+\frac{1}{k!}\bm{A}^k+\cdots$ 绝对收敛.

  结合泛函分析中定理: 赋范空间完备当且仅当绝对收敛级数必收敛. 
  故幂级数 $\bm{E}+\bm{A}+\frac{1}{2!}\bm{A}^2+\cdots+\frac{1}{k!}\bm{A}^k+\cdots$收敛,
  将其和记为 $\e^{\bm{A}}$.
\end{proof}



\begin{proposition}
  矩阵指数函数有下面的性质:
  \begin{enumerate}[1)]
    \item 若矩阵 $\bm{A}$ 和 $\bm{B}$ 是可交换的 (即 $\bm{AB}=\bm{BA}$), 则
      \[\e^{\bm{A}+\bm{B}} = \e^{\bm{A}}\e^{\bm{B}};\]
    \item 对任何矩阵 $\bm{A}$, 指数函数 $\e^{\bm{A}}$ 是可逆的, 且
      \[\bigl(\e^{\bm{A}}\bigr)^{-1} = \e^{-\bm{A}};\]
    \item 若 $\bm{P}$ 是一个非奇异的 $n$ 阶矩阵, 则
      \[\e^{\bm{P}\bm{A}\bm{P}^{-1}} = \bm{P}\e^{\bm{A}}\bm{P}^{-1}.\]
  \end{enumerate}
\end{proposition}

\begin{proof}
  (1)
  \[\begin{split}
    \e^{\bm{A}}\e^{\bm{B}}
    & = \left(\sum_{k=0}^{\infty}\frac{\bm{A}^k}{k!}\right)\left(\sum_{k=0}^{\infty}\frac{\bm{B}^k}{k!}\right)
      = \sum_{k=0}^{\infty}\sum_{i+j=k}\frac{\bm{A}^i}{i!}\frac{\bm{B}^j}{j!}
      = \sum_{k=0}^{\infty}\sum_{i=0}^k\frac{\bm{A}^i\bm{B}^{k-i}}{i!(k-i)!} \\
    & = \sum_{k=0}^{\infty}\sum_{i=0}^k\frac{1}{k!}C_k^i\bm{A}^i\bm{B}^{k-i}=\sum_{k=0}^{\infty}\frac{1}{k!}(\bm{A}+\bm{B})^k=\e^{\bm{A}+\bm{B}}.
  \end{split}\]

  (2) 由于 $\bm{A}$ 与 $-\bm{A}$ 可交换, 故
  \[\e^{\bm{A}}\e^{-\bm{A}}=\e^{\bm{0}} = \bm{E}.\]
  所以 $\left(\e^{\bm{A}}\right)^{-1}=\e^{-\bm{A}}$.

  (3)
  \[\e^{\bm{P}\bm{A}\bm{P}^{-1}}=\sum_{k=0}^{\infty}\frac{\left(\bm{P}\bm{A}\bm{P}^{-1}\right)^k}{k!}=\sum_{k=0}^{\infty}\frac{\bm{P}\bm{A}^k\bm{P}^{-1}}{k!}=\bm{P}\e^{\bm{A}}\bm{P}^{-1}.\qedhere\]
\end{proof}


本节引进新的概念: 矩阵指数函数 $\e^{\bm{A}}$, 
容易证明 $\e^{x\bm{A}}$ 是常系数齐次线性微分方程组
$\displaystyle\frac{\diff\bm{y}}{\diff x}=\bm{A}\bm{y}$ 的标准基解矩阵, 
然后利用若尔当标准型找到实际计算基解矩阵 $\e^{x\bm{A}}$ 的一个方法:
\[\e^{x\bm{A}}=\e^{x\bm{P}\bm{J}\bm{P}^{-1}}=\bm{P}\e^{x\bm{J}}\bm{P}^{-1}.\]
从而
\[\e^{x\bm{A}}\bm{P}=\bm{P}\e^{x\bm{J}}
  =\bm{P}\begin{pmatrix}\e^{x\bm{J}_1}&&&\\&\e^{x\bm{J}_2}&&\\&&\ddots&\\&&&\e^{x\bm{J}_m}
\end{pmatrix}\]
也是常系数齐次线性微分方程组的一个基解矩阵, 但是上述结果计算量较大, 将之再细致分析, 得到下面求基解矩阵的具体计算方法:

由 $|\bm{A}-\lambda\bm{E}|=0$ 求出特征值 $\lambda$, 分两种情况:

(i)若全为单根 $\lambda_1,\cdots,\lambda_n$, 
则基解矩阵为 $\bmitPhi(x)=\left(\e^{\lambda_1x}\bm{r}_1,\e^{\lambda_2x}\bm{r}_2,\cdots,\e^{\lambda_nx}\bm{r}_n\right)$, 
其中 $\bm{r}_i$ 是与 $\lambda_i$ 对应的特征向量.

(ii)若有重根, 设 $\lambda_1,\cdots,\lambda_s$ 相应的重数分别为 $n_1,\cdots,n_s$, 
则由 $(\bm{A}-\lambda_i\bm{E})^{n_i}\bm{r}=0$ 
算出 $n_i$ 个线性无关的特征向量 $\bm{r}_{10}^{(i)},\cdots,\bm{r}_{n_i0}^{(i)}$, 再由
\[\bm{r}_{jk}^{(i)}=(\bm{A}-\lambda_i\bm{E})\bm{r}_{j,k-1}^{(i)}
\quad (k=1,2,\cdots,n_i-1;j=1,2,\cdots,n_i;i=1,2,\cdots,s)\]
算出其它所需向量, 则基解矩阵为
\[\left(\e^{\lambda_1x}P_1^{(1)}(x),\cdots,\e^{\lambda_1x}P_{n_1}^{(1)}(x);\cdots;
  \e^{\lambda_sx}P_1^{(s)}(x),\cdots,\e^{\lambda_sx}P_{n_s}^{(s)}(x)\right),\]
其中 $\displaystyle P_j^{(i)}(x)=\bm{r}_{j0}^{(i)}+\frac{x}{1!}\bm{r}_{j1}^{(i)}+\frac{x^2}{2!}\bm{r}_{j2}^{(i)}+\cdots+\frac{x^{n_i-1}}{(n_i-1)!}\bm{r}_{j,n_i-1}^{(i)}$.



\subsection{习题}



\begin{exercise}
  求出常系数齐次线性微分方程组(6.25)的通解, 其中的矩阵 $\bm{A}$ 分别为:
  \begin{enumerate}
    \item $\displaystyle\begin{pmatrix}3&4\\5&2\end{pmatrix}$;
    \item $\displaystyle\begin{pmatrix}0&a\\-a&0\end{pmatrix}$;
    \item $\displaystyle\begin{pmatrix}-1&1&0\\0&-1&0\\1&0&-4\end{pmatrix}$;
    \item $\displaystyle\begin{pmatrix}-5&-10&-20\\5&5&10\\2&4&9\end{pmatrix}$;
    \item $\displaystyle\begin{pmatrix}1&\frac{2}{3}&-\frac{2}{3}\\0&\frac{2}{3}&\frac{1}{3}\\0&-\frac{1}{3}&\frac{4}{3}\end{pmatrix}$;
    \item $\displaystyle\begin{pmatrix}1&1&1&1\\1&1&-1&-1\\1&-1&1&-1\\1&-1&-1&1\end{pmatrix}$.
  \end{enumerate}
\end{exercise}

\begin{solution}
  (1) $|\bm{A}-\lambda\bm{E}|=(\lambda-7)(\lambda+2)=0\Rightarrow\lambda=7$ 或 $\lambda=-2$.

  当 $\lambda=7$ 时, $\begin{pmatrix}-4&4\\5&-5\end{pmatrix}\to\begin{pmatrix}1&-1\\0&0\end{pmatrix}$, 
  取特征向量为 $\bm{r}=\begin{pmatrix}1\\1\end{pmatrix}$.

  当 $\lambda=-2$ 时, $\begin{pmatrix}5&4\\5&4\end{pmatrix}\to\begin{pmatrix}5&4\\0&0\end{pmatrix}$, 
  取特征向量为 $\bm{r}=\begin{pmatrix}4\\-5\end{pmatrix}$.

  故基解矩阵为
  \[\bmitPhi(x)=\begin{pmatrix}\e^{7x}&4\e^{-2x}\\\e^{7x}&-5\e^{-2x}\end{pmatrix}.\]

  故通解为
  \[\bm{y}=C_1\begin{pmatrix}\e^{7x}\\\e^{7x}\end{pmatrix}
    + C_2\begin{pmatrix}4\e^{-2x}\\-5\e^{-2x}\end{pmatrix}.\]

  (2) $|\bm{A}-\lambda\bm{E}|=\lambda^2+a^2=0\Rightarrow\lambda=\pm a\upi$.

  当 $\lambda=a\upi$ 时, $\begin{pmatrix}-a\upi&a\\-a&-a\upi\end{pmatrix}\to\begin{pmatrix}-\upi&1\\0&0\end{pmatrix}$,
  取特征向量为 $\bm{r}=\begin{pmatrix}1\\\upi\end{pmatrix}$.

  当 $\lambda=-a\upi$ 时, $\begin{pmatrix}a\upi&a\\-a&a\upi\end{pmatrix}\to\begin{pmatrix}\upi&1\\0&0\end{pmatrix}$, 
  取特征向量为 $\bm{r}=\begin{pmatrix}\upi\\1\end{pmatrix}$.

  故基解矩阵为
  \[\bmitPhi(x)=\begin{pmatrix}\e^{\upi ax}&\upi\e^{-\upi ax}\\\upi\e^{\upi ax}&\e^{-\upi ax}\end{pmatrix}
  \Rightarrow\widetilde{\bmitPhi}(x)=\begin{pmatrix}\cos ax&\sin ax\\-\sin ax&\cos ax\end{pmatrix}.\]

  从而通解为
  \[\bm{y}=C_1\begin{pmatrix}\cos ax\\-\sin ax\end{pmatrix}
    + C_2\begin{pmatrix}\sin ax\\\cos ax\end{pmatrix}.\]

  (3) 由$\displaystyle\frac{\diff y_2}{\diff x}=-y_2$, 得 $y_2=C_1\e^{-x}$; 
  再由 $\displaystyle\frac{\diff y_1}{\diff x}=-y_1+y_2=-y_1+C_1\e^{-x}$, 得
  \[\displaystyle y_1=\e^{-\int\diff x}\left(C_2+\int C_1\e^{-x}\e^{\int\diff x}\diff x\right)
    = C_1x\e^{-x}+C_2\e^{-x}.\]
  再由 $\displaystyle\frac{\diff y_3}{\diff x}=y_1-4y_3=-4y_3+C_1x\e^{-x}+C_2\e^{-x}$, 得
  \[\displaystyle y_3=\e^{-\int4\diff x}\left(C_3+\int\left(C_1x\e^{-x}+C_2\e^{-x}\right)
    \e^{\int4\diff x}\diff x\right)=C_1\left(\frac{x}{3}-\frac{1}{9}\right)\e^{-x}
    +\frac{1}{3}C_2\e^{-x}+C_3\e^{-4x}.\]
  故通解为
  \[\bm{y}=C_1\begin{pmatrix}x\\1\\\frac{x}{3}-\frac{1}{9}\end{pmatrix}\e^{-x}+C_2
    \begin{pmatrix}1\\0\\\frac{1}{3}\end{pmatrix}\e^{-x}
    +C_3\begin{pmatrix}0\\0\\1\end{pmatrix}\e^{-4x}.\]

  (4) $\displaystyle|\bm{A}-\lambda\bm{E}|=\begin{vmatrix}-5-\lambda&-10&-20\\5&5-\lambda&10\\2&4&9-\lambda\end{vmatrix}=-(\lambda-5)(\lambda^2-4\lambda+5)=0\Rightarrow\lambda=5,2\pm\upi$.

  当 $\lambda=5$ 时, 
  $\begin{pmatrix}-10&-10&-20\\5&0&10\\2&4&4\end{pmatrix}\to\begin{pmatrix}1&0&2\\0&1&0\\0&0&0\end{pmatrix}$, 
  取特征向量为 $\bm{r}=\begin{pmatrix}2\\0\\-1\end{pmatrix}$.

  当 $\lambda=2+\upi$ 时, 
  $\begin{pmatrix}-7-\upi&-10&-20\\5&3-\upi&10\\2&4&7-\upi\end{pmatrix}\to\begin{pmatrix}1&0&\frac{1}{2}(3+\upi)\\0&1&\frac{1}{2}(2-\upi)\\0&0&0\end{pmatrix}$
  (这里的矩阵行变换也不是很复杂, 主要就是凑出 $1$), 
  取特征向量为 $\bm{r}=\begin{pmatrix}3+\upi\\2-\upi\\-2\end{pmatrix}$,
  故对于特征值 $\lambda=2-\upi$ 可以取到特征向量 $\bm{r}=\begin{pmatrix}3-\upi\\2+\upi\\-2\end{pmatrix}$.

  故基解矩阵为
  \[\begin{split}
  &\bmitPhi(x)=\begin{pmatrix}2\e^{5x}&(3+\upi)\e^{(2+\upi)x}&(3-\upi)\e^{(2-\upi)x}\\0&(2-\upi)\e^{(2+\upi)x}&(2+\upi)\e^{(2-\upi)x}\\-\e^{5x}&-2\e^{(2+\upi)x}&-2\e^{(2-\upi)x}\end{pmatrix}\\
  \Rightarrow&\widetilde{\bmitPhi}(x)\begin{pmatrix}2\e^{5x}&(3\cos x-\sin x)\e^{2x}&(\cos x+3\sin x)\e^{2x}\\0&(2\cos x+\sin x)\e^{2x}&(-\cos x+2\sin x)\e^{2x}\\\e^{5x}&-2\cos x\e^{2x}&-2\sin x\e^{2x}\end{pmatrix}.
  \end{split}\]
  从而通解为
  \[\bm{y} = C_1\begin{pmatrix}2\\0\\-1\end{pmatrix}\e^{5x}
    + C_2\begin{pmatrix}3\cos x-\sin x\\2\cos x+\sin x\\-2\cos x\end{pmatrix}\e^{2x}
    + C_3\begin{pmatrix}\cos x+3\sin x\\-\cos x+2\sin x\\-2\sin x\end{pmatrix}\e^{2x}.\]

  (5) $|\bm{A}-\lambda\bm{E}|=\begin{vmatrix}1-\lambda&\frac{2}{3}&-\frac{2}{3}\\0&\frac{2}{3}-\lambda&\frac{1}{3}\\0&-\frac{1}{3}&\frac{4}{3}-\lambda\end{vmatrix}=-(\lambda-1)^3=0\Rightarrow\lambda_{1,2,3}=1$, 
  由于 $(\bm{A}-\bm{E})^3=\bm{0}$, 
  故由 $(\bm{A}-\bm{E})^3\bm{r}=\bm{0}$ 得到三个线性无关的特征向量
  $\bm{r}_{10}=\begin{pmatrix}1\\0\\0\end{pmatrix},\bm{r}_{20}=\begin{pmatrix}0\\1\\0\end{pmatrix},\bm{r}_{30}=\begin{pmatrix}0\\0\\1\end{pmatrix}$, 
  且 $\bm{r}_{11}=\bm{r}_{12}=\bm{0}$, $\bm{r}_{21}=\begin{pmatrix}\frac{2}{3}\\-\frac{1}{3}\\-\frac{1}{3}\end{pmatrix}$,
  $\bm{r}_{22}=\bm{0}$, $\bm{r}_{31}=\begin{pmatrix}-\frac{2}{3}\\\frac{1}{3}\\\frac{1}{3}\end{pmatrix}$,
  $\bm{r}_{32}=\bm{0}$.

  故基解矩阵为
  \[\bmitPhi(x)=
  \begin{pmatrix}
    \e^x &\frac{2}{3}x\e^x                & -\frac{2}{3}x\e^x \\
    0    &\left(1-\frac{1}{3}x\right)\e^x & \frac{1}{3}x\e^x \\
    0    &-\frac{1}{3}x\e^x               & \left(1+\frac{1}{3}x\right)\e^x
  \end{pmatrix}.\]

  故通解为
  \[\bm{y} = C_1\begin{pmatrix}1\\0\\0\end{pmatrix}\e^x
    + C_2\begin{pmatrix}\frac{2}{3}x\\1-\frac{1}{3}x\\-\frac{1}{3}x\end{pmatrix}\e^x
    + C_3\begin{pmatrix}-\frac{2}{3}x\\\frac{1}{3}x\\1+\frac{1}{3}x\end{pmatrix}\e^x.\]

  (6) $|\bm{A}-\lambda\bm{E}|=\begin{vmatrix}1-\lambda&1&1&1\\1&1-\lambda&-1&-1\\1&-1&1-\lambda&-1\\1&-1&-1&1-\lambda\end{vmatrix}=(\lambda+2)(\lambda-2)^3=0\Rightarrow\lambda_{1,2,3}=2,\lambda_4=-2$.

  当 $\lambda=2$时, $(\bm{A}-2\bm{E})^3=\begin{pmatrix}-16&16&16&16\\16&-16&-16&-16\\16&-16&-16&-16\\16&-16&-16&-16\end{pmatrix}\to\begin{pmatrix}1&-1&-1&-1\\0&0&0&0\\0&0&0&0\\0&0&0&0\end{pmatrix}$, 由$(\bm{A}-2\bm{E})^3\bm{r}=0$
  得到三个线性无关的特征向量 $\bm{r}_{10}=\begin{pmatrix}1\\1\\0\\0\end{pmatrix}$,
  $\bm{r}_{20}=\begin{pmatrix}1\\0\\1\\0\end{pmatrix}$,
  $\bm{r}_{30}=\begin{pmatrix}1\\0\\0\\1\end{pmatrix}$, 	
  且 $\bm{r}_{ij}=\bm{0}(i=1,2,3;j=1,2)$.

  当 $\lambda=-2$时, $\bm{A}+2\bm{E}=\begin{pmatrix}3&1&1&1\\1&3&-1&-1\\1&-1&3&-1\\1&-1&-1&3\end{pmatrix}\to\begin{pmatrix}1&0&0&1\\0&1&0&-1\\0&0&1&-1\\0&0&0&0\end{pmatrix}$, 
  由 $(\bm{A}+2\bm{E})\bm{r}=\bm{0}$得特征向量$\bm{r}=\begin{pmatrix}-1\\1\\1\\1\end{pmatrix}$.

  故基解矩阵为
  \[\bmitPhi(x) = 
  \begin{pmatrix}
    \e^{2x}&\e^{2x}&\e^{2x}&-\e^{-2x} \\
    \e^{2x}&0&0&\e^{-2x} \\
    0&\e^{2x}&0&\e^{-2x} \\
    0&0&\e^{2x}&\e^{-2x}
  \end{pmatrix}.\]

  故通解为

  \[\bm{y} = 
    C_1\begin{pmatrix}1\\1\\0\\0\end{pmatrix}\e^{2x}
    + C_2\begin{pmatrix}1\\0\\1\\0\end{pmatrix}\e^{2x}+C_3\begin{pmatrix}1\\0\\0\\1\end{pmatrix}\e^{2x}
    + C_4\begin{pmatrix}-1\\1\\1\\1\end{pmatrix}\e^{-2x}.\qedhere\]
\end{solution}



\begin{exercise}
  求出常系数非齐次线性微分方程组 (6.24) 的通解, 其中:
  \begin{enumerate}[(1)]
  \item $\bm{A}=\begin{pmatrix}2&1\\0&2\end{pmatrix},
    \quad\bm{f}(x)=\begin{pmatrix}1\\0\end{pmatrix}$;
  \item $\bm{A}=\begin{pmatrix}0&-n^2\\-n^2&0\end{pmatrix},
    \quad\bm{f}(x)=\begin{pmatrix}\cos nx\\\sin nx\end{pmatrix}$;
  \item $\bm{A}=\begin{pmatrix}2&-1\\1&0\end{pmatrix},
    \quad\bm{f}(x)=\begin{pmatrix}0\\2\e^x\end{pmatrix}$;
  \item $\bm{A}=\begin{pmatrix}2&1&-2\\-1&0&0\\1&1&-1\end{pmatrix},
    \quad\bm{f}(x)=\begin{pmatrix}2-x\\0\\1-x\end{pmatrix}$;
  \item $\bm{A}=\begin{pmatrix}-1&-1&0\\0&-1&-1\\0&0&-1\end{pmatrix},
    \quad\bm{f}(x)=\begin{pmatrix}x^2\\2x\\x\end{pmatrix}$.
  \end{enumerate}
\end{exercise}

\begin{solution}
  (1) $|\bm{A}-\lambda\bm{E}|=(\lambda-2)^2=0\Rightarrow\lambda_{1,2}=2$. 
  由 $(\bm{A}-2\bm{E})^2\bm{r}=0$ 得到两个线性无关的特征向量
  $\bm{r}_{10}=\begin{pmatrix}1\\0\end{pmatrix}$,
  $\bm{r}_{20}=\begin{pmatrix}0\\1\end{pmatrix}$, 
  且 $\bm{r}_{11}=\begin{pmatrix}0\\0\end{pmatrix},\bm{r}_{21}=\begin{pmatrix}1\\0\end{pmatrix}$,
  故基解矩阵为
  \[\bmitPhi(x)=\begin{pmatrix}\e^{2x}&x\e^{2x}\\0&\e^{2x}\end{pmatrix}.\]

  通解为
  \begin{align*}
    \bm{y}
    & = \bmitPhi(x)\bm{c}+\bmitPhi(x)\int\bmitPhi^{-1}(x)\bm{f}(x)\diff x\\
    & = C_2\begin{pmatrix}1\\0\end{pmatrix}\e^{2x}
      + C_2\begin{pmatrix}x\\1\end{pmatrix}\e^{2x}-\begin{pmatrix}\frac{1}{2}\\0\end{pmatrix}.
  \end{align*}

  (2) $|\bm{A}-\lambda\bm{E}|=\lambda^2-n^4=0\Rightarrow\lambda=\pm n^2$.

  当 $\lambda=n^2$ 时, $\begin{pmatrix}-n^2&-n^2\\-n^2&-n^2\end{pmatrix}\to\begin{pmatrix}1&1\\0&0\end{pmatrix}$, 
  取特征向量为 $\bm{r}=\begin{pmatrix}1\\-1\end{pmatrix}$.

  当 $\lambda=-n^2$ 时, $\begin{pmatrix}n^2&-n^2\\-n^2&n^2\end{pmatrix}\to\begin{pmatrix}1&-1\\0&0\end{pmatrix}$,
  取特征向量为 $\bm{r}=\begin{pmatrix}1\\1\end{pmatrix}$.

  故基解矩阵为
  \[\bmitPhi(x)=\begin{pmatrix}\e^{n^2x}&\e^{-n^2x}\\-\e^{n^2x}&\e^{-n^2x}\end{pmatrix}.\]

  其逆矩阵为
  \[\bmitPhi^{-1}(x)
    = \frac{1}{2}\begin{pmatrix}\e^{-n^2x}&-\e^{-n^2x}\\\e^{n^2x}&\e^{n^2x}\end{pmatrix}.\]
  故
  \[\bmitPhi(x)^{-1}\bm{f}(x)
    = \frac{1}{2}\begin{pmatrix}\e^{-n^2x}\cos nx-\e^{-n^2x}\sin nx\\\e^{n^2x}\cos nx+\e^{n^2x}\sin nx\end{pmatrix}.\]
  利用不定积分公式
  \[\begin{cases}\displaystyle
  \int\e^{\alpha x}\sin\beta x\diff x=\frac{\alpha}{\alpha^2+\beta^2}\e^{\alpha x}\sin\beta x-\frac{\beta}{\alpha^2+\beta^2}\e^{\alpha x}\cos\beta x+C, \\
  \displaystyle\int\e^{\alpha x}\cos\beta x\diff x=\frac{\alpha}{\alpha^2+\beta^2}\e^{\alpha x}\cos\beta x+\frac{\beta}{\alpha^2+\beta^2}\e^{\alpha x}\sin\beta x+C
  \end{cases}\]
  得
  \[\int\bmitPhi(x)^{-1}\bm{f}(x)\diff x=\frac{1}{2}\begin{pmatrix}\frac{n+1}{n^3+n}\e^{-n^2x}\sin nx+\frac{-n+1}{n^3+n}\e^{-n^2x}\cos nx\\
  \frac{n+1}{n^3+n}\e^{n^2x}\sin nx+\frac{n-1}{n^3+n}\e^{n^2x}\cos nx
  \end{pmatrix}.\]

  故
  \[\bmitPhi(x)\int\bmitPhi(x)^{-1}\bm{f}(x)\diff x
    =\begin{pmatrix}
      \frac{n+1}{n^3+n}\sin nx \\
      \frac{n-1}{n^3+n}\cos nx
     \end{pmatrix}.\]

  因此通解为
  \[\bm{y} = C_1\begin{pmatrix}1\\-1\end{pmatrix}\e^{n^2x}
    + C_2\begin{pmatrix}1\\1\end{pmatrix}\e^{-n^2x}
    + \begin{pmatrix}
        \frac{n+1}{n^3+n}\sin nx\\
        \frac{n-1}{n^3+n}\cos nx
      \end{pmatrix}.\]

  (3) $|\bm{A}-\lambda\bm{E}|=\begin{vmatrix}2-\lambda&-1\\1&-\lambda\end{vmatrix}=(\lambda-1)^2=0\Rightarrow\lambda=1$,
  由 $(\bm{A}-\bm{E})^2\bm{r}=0$ 得到两个线性无关得特征向量 
  $\bm{r}_{10}=\begin{pmatrix}1\\0\end{pmatrix},\bm{r}_{20}=\begin{pmatrix}0\\1\end{pmatrix}$, 
  且 $\bm{r}_{11}=\begin{pmatrix}1\\1\end{pmatrix},\bm{r}_{21}=\begin{pmatrix}-1\\-1\end{pmatrix}$.
  故基解矩阵为
  \[\bmitPhi(x)=\begin{pmatrix}(x+1)\e^x&-x\e^x\\x\e^x&(1-x)\e^x\end{pmatrix}.\]
  故
  \[\bmitPhi(x)\int\bmitPhi(x)^{-1}\bm{f}(x)\diff x
    = \begin{pmatrix}-x^2\e^x\\(-x^2+2x)\e^x\end{pmatrix}.\]
  故通解为
  \[\bm{y} = C_1\begin{pmatrix}x+1\\x\end{pmatrix}\e^x
    + C_2\begin{pmatrix}-x\\-x+1\end{pmatrix}\e^x
    + \begin{pmatrix}-x^2\e^x\\(-x^2+2x)\e^x\end{pmatrix}.\]

  (4) $|\bm{A}-\lambda\bm{E}|=\begin{vmatrix}2-\lambda&1&-2\\-1&-\lambda&0\\1&1&-1-\lambda\end{vmatrix}=-(\lambda-1)(\lambda^2+1)=0\Rightarrow\lambda=1,\pm\upi$.

  当 $\lambda=1$ 时, $\begin{pmatrix}1&1&-2\\-1&-1&0\\1&1&-2\end{pmatrix}\to\begin{pmatrix}1&1&0\\0&0&1\\0&0&0\end{pmatrix}$, 
  取特征向量为 $\bm{r}=\begin{pmatrix}1\\-1\\0\end{pmatrix}$.

  当 $\lambda=\upi$ 时, $\begin{pmatrix}2-\upi&1&-2\\-1&-\upi&0\\1&1&-1-\upi\end{pmatrix}\to\begin{pmatrix}1&0&-1\\0&1&\upi\\0&0&0\end{pmatrix}$, 
  取特征向量为 $\bm{r}=\begin{pmatrix}1\\\upi\\1\end{pmatrix}$.

  当 $\lambda=-\upi$ 时, 可取特征向量为 $\bm{r}=\begin{pmatrix}\upi\\1\\\upi\end{pmatrix}$
  (注意共轭特征向量为共轭特征值的特征向量).

  故基解矩阵为
  \[\widetilde{\bmitPhi}(x) = 
  \begin{pmatrix}
    \e^x&\e^{\upi x}&\upi\e^{-\upi x} \\
    -\e^x&\upi\e^{\upi x}&\e^{-\upi x} \\
    0&\e^{\upi x}&\upi\e^{-\upi x}
  \end{pmatrix}\Rightarrow
  \bmitPhi(x) =
  \begin{pmatrix}
    \e^x&\cos x&\sin x \\
    -\e^x&-\sin x&\cos x \\
    0&\cos x&\sin x
  \end{pmatrix}.\]
  故
  \[\bmitPhi^{-1}(x) =
  \begin{pmatrix}
    \e^{-x}&0&-\e^{-x} \\
    -\sin x&-\sin x&\sin x+\cos x \\
    \cos x&\cos x&\sin x-\cos x
  \end{pmatrix}.\]
  故
  \[\bmitPhi^{-1}(x)\bm{f}(x) =
  \begin{pmatrix}
    \e^{-x} \\
    (1-x)\cos x-\sin x \\
    (1-x)\sin x+\cos x
  \end{pmatrix}\Rightarrow
  \int\bmitPhi^{-1}(x)\bm{f}(x)\diff x =
  \begin{pmatrix}
    -\e^{-x} \\ 
    (1-x)\sin x \\
    (x-1)\cos x
  \end{pmatrix}.\]
  故
  \[\bmitPhi(x)\int\bmitPhi^{-1}(x)\bm{f}(x)\diff x
    = \begin{pmatrix}-1\\x\\0\end{pmatrix}.\]
  故通解为
  \[\bm{y} = C_1\begin{pmatrix}1\\-1\\0\end{pmatrix}\e^x
    + C_2\begin{pmatrix}\cos x\\-\sin x\\\cos x\end{pmatrix}
    + C_3\begin{pmatrix}\sin x\\\cos x\\\sin x\end{pmatrix}
    + \begin{pmatrix}-1\\x\\0\end{pmatrix}.\]

  (5) $|\bm{A}-\lambda\bm{E}|=-(\lambda+1)^3=0\Rightarrow\lambda_{1,2,3}=-1$.

  由 $(\bm{A}+\bm{E})^3\bm{r}=0$ 得到三个线性无关的特征向量
  $\bm{r}_{10}=\begin{pmatrix}1\\0\\0\end{pmatrix}$,
  $\bm{r}_{20}=\begin{pmatrix}0\\1\\0\end{pmatrix}$,
  $\bm{r}_{30}=\begin{pmatrix}0\\0\\1\end{pmatrix}$, 
  且 $\bm{r}_{11}=\bm{r}_{12}=\bm{r}_{22}=\bm{0}$,
  $\bm{r}_{21}=\begin{pmatrix}-1\\0\\0\end{pmatrix}$,
  $\bm{r}_{31}=\begin{pmatrix}0\\-1\\0\end{pmatrix}$,
  $\bm{r}_{32}=\begin{pmatrix}1\\0\\0\end{pmatrix}$.

  故基解矩阵为
  \[\bmitPhi(x) = 
  \begin{pmatrix}
    \e^{-x}&-x\e^{-x}&\frac{1}{2}x^2\e^{-x} \\
    0&\e^{-x}&-x\e^{-x} \\
    0&0&\e^{-x}
  \end{pmatrix}.\]
  故
  \[\bmitPhi^{-1}(x) = 
  \begin{pmatrix}
    \e^x&x\e^x&\frac{1}{2}x^2\e^x \\
    0&\e^x&x\e^x \\
    0&0&\e^x
  \end{pmatrix}.\]
  故
  \[\bmitPhi^{-1}(x)\bm{f}(x) =
  \begin{pmatrix}
    3x^2\e^x+\frac{1}{2}x^3\e^x \\
    2x\e^x+x^2\e^x \\
    x\e^x
  \end{pmatrix}\Rightarrow
  \int\bmitPhi^{-1}(x)\bm{f}(x)\diff x =
  \begin{pmatrix}
    \left(\frac{1}{2}x^3+\frac{3}{2}x^2-3x+3\right)\e^x \\
    x^2\e^x \\
    (x-1)\e^x
  \end{pmatrix}.\]
  故
  \[\bmitPhi(x)\int\bmitPhi^{-1}(x)\bm{f}(x)\diff x=\begin{pmatrix}x^2-3x+3\\x\\x-1\end{pmatrix}.\]

  故通解为
  \[\bm{y} = C_1\begin{pmatrix}1\\0\\0\end{pmatrix}\e^{-x}
    + C_2\begin{pmatrix}-x\\1\\0\end{pmatrix}\e^{-x}
    + C_3\begin{pmatrix}x^2\\-2x\\2\end{pmatrix}\e^{-x}
    + \begin{pmatrix}x^2-3x+3\\x\\x-1\end{pmatrix}.\qedhere\]
\end{solution}



\begin{exercise}
  求出微分方程组 (6.24) 满足初值条件 $\bm{y}(0)=\bm{\eta}$ 的解, 其中:
  \begin{enumerate}[(1)]
  \item $\bm{A}=\begin{pmatrix}-5&-1\\1&-3\end{pmatrix},
    \quad\bm{f}(x)=\begin{pmatrix}\e^x\\\e^{2x}\end{pmatrix},
    \quad\bm{\eta}=\begin{pmatrix}1\\0\end{pmatrix}$;
  \item $\bm{A}=\begin{pmatrix}0&-2\\2&0\end{pmatrix},
    \quad\bm{f}(x)=\begin{pmatrix}3x\\4\end{pmatrix},
    \quad\bm{\eta}=\begin{pmatrix}2\\3\end{pmatrix}$;
  \item $\bm{A}=\begin{pmatrix}4&-3\\2&-1\end{pmatrix},
    \quad\bm{f}(x)=\begin{pmatrix}\sin x\\-2\cos x\end{pmatrix},
    \quad\bm{\eta}=\begin{pmatrix}0\\0\end{pmatrix}$;
  \item $\bm{A}=\begin{pmatrix}16&14&38\\-9&-7&-18\\-4&-4&-11\end{pmatrix},
    \quad\bm{f}(x)=\begin{pmatrix}-2\e^{-x}\\-3\e^{-x}\\2\e^{-x}\end{pmatrix},
    \quad\bm{\eta}=\begin{pmatrix}0\\0\\0\end{pmatrix}$.
  \end{enumerate}
\end{exercise}

\begin{solution}
  (1) $|\bm{A}-\lambda\bm{E}|=(\lambda+4)^2=0\Rightarrow\lambda_{1,2}=-4$.

  由 $(\bm{A}+4\bm{E})^2\bm{r}=0$ 得两个线性无关的特征向量
  $\bm{r}_{10}=\begin{pmatrix}1\\0\end{pmatrix}$,
  $\bm{r}_{20}=\begin{pmatrix}0\\1\end{pmatrix}$, 
  且 $\bm{r}_{11}=\begin{pmatrix}-1\\1\end{pmatrix},\bm{r}_{21}=\begin{pmatrix}-1\\1\end{pmatrix}$.

  故基解矩阵为
  \[\bmitPhi(x)=\begin{pmatrix}(1-x)\e^{-4x}&-x\e^{-4x}\\x\e^{-4x}&(1+x)\e^{-4x}\end{pmatrix}.\]
  故
  \[\bmitPhi^{-1}(x)=\begin{pmatrix}(1+x)\e^{4x}&x\e^{4x}\\-x\e^{4x}&(1-x)\e^{4x}\end{pmatrix}.\]
  故
  \[\bmitPhi^{-1}(x)\bm{f}(x)=\begin{pmatrix}(1+x)\e^{5x}+x\e^{6x}\\-x\e^{5x}+(1-x)\e^{6x}\end{pmatrix}\Rightarrow\int\bmitPhi^{-1}(x)\bm{f}(x)\diff x=
  \renewcommand\arraystretch{1.2}\begin{pmatrix}\frac{4}{25}\e^{5x}+\frac{1}{5}x\e^{5x}-\frac{1}{36}\e^{6x}+\frac{1}{6}x\e^{6x}\\-\frac{1}{5}x\e^{5x}+\frac{1}{25}\e^{5x}+\frac{7}{36}\e^{6x}-\frac{1}{6}x\e^{6x}\end{pmatrix}.\]
  故
  \[\bmitPhi(x)\int\bmitPhi^{-1}(x)\bm{f}(x)\diff x=
  \renewcommand\arraystretch{1.2}\begin{pmatrix}\frac{4}{25}\e^x-\frac{1}{36}\e^{2x}\\\frac{1}{25}\e^x+\frac{7}{36}\e^{2x}\end{pmatrix}.\]
  故通解为
  \[\bm{y}=C_1\begin{pmatrix}(1-x)\e^{-4x}\\x\e^{-4x}\end{pmatrix}+C_2\begin{pmatrix}-x\e^{-4x}\\(1+x)\e^{-4x}\end{pmatrix}+\renewcommand\arraystretch{1.2}\begin{pmatrix}\frac{4}{25}\e^x-\frac{1}{36}\e^{2x}\\\frac{1}{25}\e^x+\frac{7}{36}\e^{2x}\end{pmatrix}.\]
  结合初值条件
  \[C_1\begin{pmatrix}1\\0\end{pmatrix}+C_2\begin{pmatrix}0\\1\end{pmatrix}+\begin{pmatrix}\frac{4}{25}-\frac{1}{36}\\\frac{1}{25}+\frac{7}{36}\end{pmatrix}=\begin{pmatrix}1\\0\end{pmatrix}\]
  得 $C_1=\frac{781}{900},C_2=\frac{-211}{900}$, 故初值问题的解为
  \[\bm{y} = 
    \frac{781}{900}\begin{pmatrix}1-x\\x\end{pmatrix}\e^{-4x}
    + \frac{-211}{900}\begin{pmatrix}-x\\1+x\end{pmatrix}\e^{-4x}
    + \begin{pmatrix}
      \frac{4}{25}\e^x-\frac{1}{36}\e^{2x} \\
      \frac{1}{25}\e^x+\frac{7}{36}\e^{2x}
    \end{pmatrix}.\]

  (2) $|\bm{A}-\lambda\bm{E}|=\lambda^2+4=0\Rightarrow\lambda=\pm2\upi$.

  当 $\lambda=2\upi$ 时, $\begin{pmatrix}-2\upi&-2\\2&-2\upi\end{pmatrix}\to\begin{pmatrix}1&-\upi\\0&0\end{pmatrix}$, 
  取特征向量为 $\bm{r}=\begin{pmatrix}\upi\\1\end{pmatrix}$.

  当 $\lambda=-2\upi$ 时, 取特征向量为 $\bm{r}=\begin{pmatrix}1\\\upi\end{pmatrix}$.

  故基解矩阵为
  \[\widetilde{\bmitPhi}(x) =
    \begin{pmatrix}
      \upi\e^{2\upi x}&\e^{-2\upi x} \\
      \e^{2\upi x}&\upi\e^{-2\upi x}
    \end{pmatrix}\Rightarrow\bmitPhi(x) =
    \begin{pmatrix}\cos2x&-\sin2x\\\sin2x&\cos2x\end{pmatrix}.\]
  故
  \[\bmitPhi^{-1}(x)=\begin{pmatrix}\cos2x&\sin2x\\-\sin2x&\cos2x\end{pmatrix}.\]
  故
  \[\bmitPhi^{-1}(x)\bm{f}(x) = 
    \begin{pmatrix}3x\cos2x+4\sin2x\\-3x\sin2x+4\cos2x\end{pmatrix}
    \Rightarrow
    \int\bmitPhi^{-1}(x)\bm{f}(x)\diff x =
    \begin{pmatrix}
      \frac{3}{2}x\sin2x-\frac{5}{4}\cos2x \\
      \frac{3}{2}x\cos2x+\frac{5}{4}\sin2x
    \end{pmatrix}.\]
  故
  \[\bmitPhi(x)\int\bmitPhi^{-1}(x)\bm{f}(x)\diff x =
    \begin{pmatrix}-\frac{5}{4}\\\frac{3}{2}x\end{pmatrix}.\]
  故通解为
  \[\bm{y} = C_1\begin{pmatrix}\cos2x\\\sin2x\end{pmatrix}
    + C_2\begin{pmatrix}-\sin2x\\\cos2x\end{pmatrix}
    + \begin{pmatrix}-\frac{5}{4}\\\frac{3}{2}x\end{pmatrix}.\]
  结合初值条件
  \[C_1\begin{pmatrix}1\\0\end{pmatrix}
    + C_2\begin{pmatrix}0\\1\end{pmatrix}
    + \begin{pmatrix}-\frac{5}{4}\\0\end{pmatrix}
    = \begin{pmatrix}2\\3\end{pmatrix},\]
  得 $C_1=\frac{13}{4},C_2=3$, 故初值问题的解为
  \[\bm{y} =
    \frac{13}{4}\begin{pmatrix}\cos2x\\\sin2x\end{pmatrix}
    + 3\begin{pmatrix}-\sin2x\\\cos2x\end{pmatrix}
    + \begin{pmatrix}-\frac{5}{4}\\\frac{3}{2}x\end{pmatrix}.\]

  (3) $|\bm{A}-\lambda\bm{E}|=(\lambda-1)(\lambda-2)=0\Rightarrow\lambda_1=1,\lambda_2=2$.

  当 $\lambda=1$ 时, $\begin{pmatrix}3&-3\\2&-2\end{pmatrix}\to\begin{pmatrix}1&-1\\0&0\end{pmatrix}$, 
  取特征向量为 $\bm{r}=\begin{pmatrix}1\\1\end{pmatrix}$.

  当 $\lambda=2$ 时, $\begin{pmatrix}2&-3\\2&-3\end{pmatrix}\to\begin{pmatrix}2&-3\\0&0\end{pmatrix}$, 
  取特征向量为 $\bm{r}=\begin{pmatrix}3\\2\end{pmatrix}$.

  故基解矩阵为
  \[\bmitPhi(x)=\begin{pmatrix}\e^x&3\e^{2x}\\\e^x&2\e^{2x}\end{pmatrix}.\]
  故
  \[\bmitPhi^{-1}(x)=\begin{pmatrix}-2\e^{-x}&3\e^{-x}\\\e^{-2x}&-\e^{-2x}\end{pmatrix}.\]
  故
  \[\bmitPhi^{-1}(x)\bm{f}(x) =
    \begin{pmatrix}-2\e^{-x}\sin x-6\e^{-x}\cos x\\\e^{-2x}\sin x+2\e^{-2x}\cos x\end{pmatrix}\]
  积分得
  \[\int\bmitPhi^{-1}(x)\bm{f}(x)\diff x =
    \begin{pmatrix}-2\e^{-x}\sin x+4\e^{-x}\cos x\\-2\e^{-2x}\cos x\end{pmatrix}.\]
  故
  \[\bmitPhi(x)\int\bmitPhi^{-1}(x)\bm{f}(x)\diff x =
    \begin{pmatrix}-2\sin x+\cos x\\-2\sin x+2\cos x\end{pmatrix}.\]
  故通解为
  \[\bm{y} = C_1\begin{pmatrix}1\\1\end{pmatrix}\e^x
    + C_2\begin{pmatrix}3\\2\end{pmatrix}\e^{2x}
    + \begin{pmatrix}-2\sin x+\cos x\\-2\sin x+2\cos x\end{pmatrix}.\]
  结合初值条件
  \[C_1\begin{pmatrix}1\\1\end{pmatrix}
    + C_2\begin{pmatrix}3\\2\end{pmatrix}
    + \begin{pmatrix}1\\2\end{pmatrix}
    = \begin{pmatrix}0\\0\end{pmatrix},\]
  得 $C_1=-4$, $C_2=1$, 故初值问题的解为
  \[\bm{y} =
    - 4\begin{pmatrix}1\\1\end{pmatrix}\e^x
    + \begin{pmatrix}3\\2\end{pmatrix}\e^{2x}
    + \begin{pmatrix}-2\sin x+\cos x\\-2\sin x+2\cos x\end{pmatrix}.\]
  (4) $(-2x,-3x,2x)\e^{-x}$.
\end{solution}



\begin{exercise}
  求解微分方程组
  \[\frac{\diff}{\diff t}\begin{pmatrix}x\\y\end{pmatrix}
    = \begin{pmatrix}a&-b\\b&a\end{pmatrix}\begin{pmatrix}x\\y\end{pmatrix},\]
  其中 $a$ 和 $b$ 为实常数, 而且 $b\neq 0$.
\end{exercise}

\begin{solution}
  $\begin{vmatrix}a-\lambda&-b\\b&a-\lambda\end{vmatrix}=(a-\lambda)^2+b^2=0\Rightarrow\lambda=a\pm b\upi$.

  当 $\lambda=a+b\upi$ 时, 
  $\begin{pmatrix}-b\upi&-b\\b&-b\upi\end{pmatrix}\to\begin{pmatrix}\upi&1\\0&0\end{pmatrix}$, 
  取特征向量为 $\bm{r}=\begin{pmatrix}\upi\\1\end{pmatrix}$.

  当 $\lambda=a-b\upi$ 时, 取特征向量为 $\bm{r}=\begin{pmatrix}1\\\upi\end{pmatrix}$.

  故基解矩阵为
  \[\widetilde{\bmitPhi}(t) = 
    \begin{pmatrix}\upi\e^{(a+b\upi)t}&\e^{(a-b\upi)t}\\\e^{(a+b\upi)t}&\upi\e^{(a-b\upi)t}\end{pmatrix}
    \Rightarrow\bmitPhi(t) =
    \begin{pmatrix}\e^{at}\cos bt&-\e^{at}\sin bt\\ \e^{at}\sin bt&\e^{at}\cos bt\end{pmatrix}.\]
  故通解为
  \[\begin{pmatrix}x\\y\end{pmatrix} =
    C_1\begin{pmatrix}\e^{at}\cos bt\\\e^{at}\sin bt\end{pmatrix}
    + C_2\begin{pmatrix}-\e^{at}\sin bt\\\e^{at}\cos bt\end{pmatrix}.\qedhere\]
\end{solution}



\begin{exercise}
  证明: 常系数齐次线性微分方程组 $\displaystyle\frac{\diff\bm{y}}{\diff x}=\bm{A}\bm{y}$
  的任何解当 $x\to+\infty$ 时都趋于零, 当且仅当它的系数矩阵 $\bm{A}$ 的所有特征根都具有负的实部.
\end{exercise}

\begin{proof}
  方程的基解矩阵为
  \[\left(\e^{\lambda_1x}P_1^{(1)}(x),\cdots,\e^{\lambda_1x}P_{n_1}^{(1)}(x);\cdots;
    \e^{\lambda_sx}P_1^{(s)}(x),\cdots,\e^{\lambda_sx}P_{n_s}^{(s)}(x)\right).\]
  故
  \[\begin{split}
       & \text{当\ }x\to+\infty\text{\ 时任何解都趋于零} \\
  \iff & \e^{\lambda_ix}\to0\;(x\to+\infty)\quad (i=1,2,\cdots,s)\\
  \iff & \Re(\lambda_i)<0\quad (i=1,2,\cdots,s).\qedhere
  \end{split}\]
\end{proof}



\section{高阶线性微分方程}



\subsection{证明与总结}



对于高阶线性微分方程
\[y^{(n)}+a_1(x)y^{(n-1)}+\cdots+a_{n-1}(x)y'+a_n(x)y=f(x),\]
令 $y_1=y,y_2=y',\cdots,y_n=y^{(n-1)}$, 记 $\bm{y}=(y_1,\cdots,y_n)^{\T}$ 则
\[\frac{\diff\bm{y}}{\diff x}
  = \bm{A}(x)\bm{y}+\bm{f}(x)
  = \begin{pmatrix}
      0&1&0&\cdots&0 \\
      0&0&1&\cdots&0 \\
      \vdots&\vdots&\vdots&&\vdots \\
      0&0&0&\cdots&1 \\
      -a_n(x)&-a_{n-1}(x)&-a_{n-2}(x)&\cdots&-a_1(x)
    \end{pmatrix}
    \begin{pmatrix}
      y_1\\y_2\\\vdots\\y_n
    \end{pmatrix}+\begin{pmatrix}0\\0\\\vdots\\f(x)\end{pmatrix}.\]

(I)齐次方程: 有 $n$ 个线性无关的解 $\varphi_1(x),\cdots,\varphi_n(x)$, 解组的朗斯基行列式
\[W(x)=W(x_0)\e^{-\int_{x_0}^xa_1(s)\diff s}.\]
(一个运用: 二阶齐次线性微分方程组 $y''+p(x)y'+q(x)y=0$ 若知道一个特解可以求出通解).

(II)非齐次方程: 通解为
\[y=C_1\varphi_1(x)+\cdots+C_n\varphi_n(x)+\varphi^*(x),\] 
其中特解
\[\varphi^*(x)=\sum_{k=1}^n\varphi_k(x)\int_{x_0}^x\frac{W_k(s)}{W(s)}f(s)\diff s.\] 
这里 $W(x)$ 是 $\varphi_1(x),\cdots,\varphi_n(x)$
的 Wronsky 行列式, 而 $W_k(x)$ 是 $W(x)$ 中第 $n$ 行第 $k$ 列元素的代数余子式.
此公式既可以用前面的公式 
\[\bm{y}=\bmitPhi(x)\left(\bm{c}+\int_{x_0}^x\bmitPhi^{-1}(s)\bm{f}(s)\diff s\right)\]
取第一个分量导出, 也可以用常数变易法导出 (见习题 6).

esp: 常系数高阶线性微分方程, 其系数矩阵为
\[\bm{A} = 
  \begin{pmatrix}
    0 & 1 & 0 & \cdots & 0 \\
    0 & 0 & 1 & \cdots & 0 \\
    \vdots&\vdots&\vdots&&\vdots \\
    0 & 0 & 0 & \cdots & 1 \\
    -a_n&-a_{n-1}&-a_{n-2}&\cdots&-a_1
  \end{pmatrix}.\]
由 $|\bm{A}-\lambda\bm{E}|=\lambda^n+a_1\lambda^{n-1}+\cdots+a_{n-1}\lambda+a_n=0$
算出特征值 $\lambda_1,\lambda_2,\cdots,\lambda_s$, 其重数分别为$n_1,n_2,\cdots,n_s$
$(n_1+n_2+\cdots+n_s=n)$, 则齐次方程基本解组为:
\[\begin{cases}
\e^{\lambda_1x},x\e^{\lambda_1x},\cdots,x^{n_1-1}\e^{\lambda_1x};\\
\cdots\cdots\cdots\\
\e^{\lambda_sx},x\e^{\lambda_sx},\cdots,x^{n_s-1}\e^{\lambda_sx}.
\end{cases}\]

对于非齐次方程, 还需要求出特解, 一般用上述特解求解公式, 在 $f(x)$ 形式特殊时, 可以用待定系数法:

当 $f(x)=P_m(x)\e^{\mu x}$ 时, 取
\[\varphi^*(x)=x^kQ_m(x)\e^{\mu x},\]
其中 $\mu$ 为 $k$ 重特征根.

当 $f(x)=[A_m(x)\cos(\beta x)+B_l(x)\sin(\beta x)]\e^{\alpha x}$ 时, 取
\[\varphi^*(x)=x^k[C_n(x)\cos(\beta x)+D_n(x)\sin(\beta x)]\e^{\alpha x},\]
其中 $\alpha\pm\upi\beta$ 为 $k$ 重特征根, $n=\max\{m,l\}$.



\subsection{习题}



\begin{exercise}
  证明函数组
  \[\varphi_1(x) =
    \begin{cases}
      x^2, & \text{当\ } x\geq 0, \\
      0,   & \text{当\ } x<0;
    \end{cases}\quad
    \varphi_2(x)=
    \begin{cases}
      0,   & \text{当\ } x\geq 0, \\
      x^2, & \text{当\ } x<0
    \end{cases}\]
  在区间 $(-\infty,+\infty)$ 上线性无关, 但它们的朗斯基行列式恒等于零. 
  这与本节的定理 $6.2^*$ 是否矛盾? 如果并不矛盾, 那么它说明了什么?
\end{exercise}

\begin{proof} 
  设 $k_1\varphi_1(x)+k_2\varphi_2(x)=0, \forall x\in (-\infty,+\infty)$,
  当 $x\geq 0$时, $k_1x^2=0\Rightarrow k_1=0$, 
  当 $x<0$ 时, $k_2x^2=0\Rightarrow k_2=0$, 
  故 $\varphi_1(x)$ 与 $\varphi_2(x)$ 在 $(-\infty,+\infty)$ 上线性无关. 
  这与定理 $6.2^*$ 不矛盾, 并说明不存在二阶齐次线性微分方程使得它以 $\varphi_1(x)$ 与 $\varphi_2(x)$ 为解组.
\end{proof}



\begin{exercise}
  证明命题5.
\end{exercise}

\begin{proof} 
  ($\Rightarrow$)显然

  ($\Leftarrow$) 设 $k_1\varphi_1(x)+\cdots+k_n\varphi_n(x)=0$, 则
  \[k_1\varphi_1'(x)+\cdots+k_n\varphi_n'(x)=0,\]
  \[\cdots\]
  \[k_1\varphi_1^{(n-1)}(x)+\cdots+k_n\varphi_n^{(n-1)}(x)=0.\]
  由向量函数组线性无关即得 $k_i=0$ $(i=1,2,\cdots,n)$, 故 $\varphi_1(x),\cdots,\varphi_n(x)$ 线性无关.
\end{proof}



\begin{exercise}
  考虑微分方程: $y''+q(x)y=0$.
  \begin{enumerate}[(1)]
  \item 设 $y=\varphi(x)$ 与 $y=\psi(x)$ 是它的两个解, 
    试证 $\varphi(x)$ 与 $\psi(x)$ 的朗斯基行列式恒等于一个常数.
  \item 设已知方程有一个特解为 $y=\e^x$, 试求这方程的通解, 并确定 $q(x)=$ ?
  \end{enumerate}
\end{exercise}

\begin{solution}
  (1) $W(x)=W(x_0)\e^{-\int_{x_0}^x0\diff s}=W(x_0)$.

  (2) 将 $y=\e^x$ 代入原方程得 $q(x)=-1$, 即原方程为 $y''-y=0$, 解得通解为 $y=C_1\e^x+C_2\e^{-x}$.
\end{solution}



\begin{exercise}
  考虑微分方程
  \begin{equation}
    y''+p(x)y'+q(x)=0, \tag{$\star$}
  \end{equation}
  其中 $p(x)$ 和 $q(x)$ 是区间 $I:a<x<b$ 上的连续函数.
  \begin{enumerate}[(1)]
  \item 设 $y=\varphi(x)$ 是方程 $(\star)$ 在区间 $I$ 上的一个非零解
  (即 $\varphi(x)$ 在区间 $I$ 上不恒等于零), 
  试证 $\varphi(x)$ 在区间 $I$ 上只有简单零点
  (即: 如果存在 $x_0\in I$, 使得 $\varphi(x_0)=0$, 那么必有 $\varphi'(x_0)\neq 0$). 
  并由此进一步证明, $\varphi(x)$ 在任意有限闭区间上至多有有限个零点, 从而每一个零点都是孤立的.
  \item 在例 1 中, 对一般的情形证明相应的结论.
  \end{enumerate}
\end{exercise}

\begin{solution}
  (1) 假设存在 $x_0\in I$ 使得 $\varphi(x_0)=\varphi'(x_0)=0$, 
  则由解的存在唯一性定理知方程只有零解 $y=\varphi(x)=0$, 与 $\varphi(x)$ 是非零解相矛盾, 
  故 $\varphi(x)$ 在区间 $I$ 上只有简单零点.

  设 $J$ 是 $I$ 中有限闭区间, 且 $\varphi(x)$ 在区间 $J$ 上有无限个零点, 记为 $\{x_n\}_{n\geq 1}$, 
  由 Bolzano-Weierstrass 定理知 $\{x_n\}_{n\geq 1}$ 有收敛子列,
  不妨就设其本身收敛且 $\lim\limits_{n\to\infty}x_n=x_0\in J$, 由连续性知 $\varphi(x_0)=0$, 故
  \[\varphi'(x_0)=\lim_{x_n\to x_0}\frac{\varphi(x_n)-\varphi(x_0)}{x_n-x_0}=0.\]
  由存在唯一性定理知 $\varphi(x)\equiv 0$, 矛盾, 故 $\varphi(x)$ 的每一个零点都是孤立的.

  (2) 情形 1: $\varphi(x)$ 在区间 $I$ 上恒不为零, 设 $y=y(x)$ 是方程的任意一个解, 则由刘维尔公式得
  \[\begin{vmatrix}\varphi&y\\\varphi'&y'\end{vmatrix}
    = \varphi y'-\varphi'y=C_2\e^{-\int_{x_0}^xp(t)\diff t}.\]
  在上式两边同时乘以 $\frac{1}{\varphi^2}$, 则得
  \[\frac{\diff}{\diff x}\left(\frac{y}{\varphi}\right)
    = \frac{C_2}{\varphi^2}\e^{-\int_{x_0}^xp(t)\diff t}.\]
  将上式从 $x_0$ 到 $x$ 积分得
  \[\int_{x_0}^x\frac{\diff}{\diff s}\left(\frac{y}{\varphi}\right)\diff s
    = \frac{y(x)}{\varphi(x)}-C_1
    = \int_{x_0}^x\frac{C_2}{\varphi^2(s)}\e^{-\int_{x_0}^sp(t)\diff t}\diff s.\]
  故
  \[y(x) = \varphi(x)\left[C_1+C_2\int_{x_0}^x\frac{1}{\varphi^2(s)}\e^{-\int_{x_0}^sp(t)\diff t}\diff s\right],\]
  其中 $C_1,C_2$ 是任意常数.

  情形2: $\varphi(x)$ 是非零解, 由 (1) 知 $\varphi(x)$ 的每一个零点都是孤立的, 
  利用 $\varphi(x)$ 的零点将区间 $(a,b)$ 分割为开区间之并.
\end{solution}



\begin{exercise}
  设函数 $u(x)$ 和 $v(x)$ 是方程 $y''+p(x)y'+q(x)y=0$ 的一个基本解组, 试证:
  \begin{enumerate}[(1)]
  \item 方程的系数函数 $p(x)$ 和 $q(x)$ 能由这个基本解组唯一地确定.
  \item $u(x)$ 和 $v(x)$ 没有共同的零点.
  \end{enumerate}
\end{exercise}

\begin{proof}
  (1) 依题意得 $\begin{cases}p(x)u'+q(x)u=-u''\\p(x)v'+q(x)v=-v''\end{cases}$, 
  又 $\begin{vmatrix}u'&u\\v'&v\end{vmatrix}=-W[u(x),v(x)]\neq 0$, 故
  \[p(x) 
    = \frac{\begin{vmatrix}-u''&u\\-v''&v\end{vmatrix}}{\begin{vmatrix}u'&u\\v'&v\end{vmatrix}}
    = \frac{u''v-uv''}{W[u(x),v(x)]},\quad
    q(x)
    = \frac{\begin{vmatrix}u'&-u''\\v'&-v''\end{vmatrix}}{\begin{vmatrix}u'&u\\v'&v\end{vmatrix}}
    = \frac{u'v''-u''v'}{W[u(x),v(x)]}.\]
  也即 $p(x)$ 和 $q(x)$ 能由这个基本解组唯一地确定.

  (2) 假设 $u(x)$ 和 $v(x)$ 有共同的零点 $x_0$, 则 $W[u(x_0),v(x_0)]=0$, 
  与 $u(x)$ 和 $v(x)$ 为基本解组相矛盾, 故 $u(x)$ 和 $v(x)$ 没有共同的零点.
\end{proof}



\begin{exercise}
  试用常数变易法证明定理 $6.3^*$.
\end{exercise}

\begin{proof} 
  设非齐次线性微分方程的通解为
  \[y=C_1(x)\varphi_1(x)+\cdots+C_n(x)\varphi_n(x).\]
  则
  \[y'=C_1'(x)\varphi_1(x)+\cdots+C_n'(x)\varphi_n(x)+C_1(x)\varphi_1'(x)+\cdots+C_n(x)\varphi_n'(x).\]
  令 $C_1'(x)\varphi_1(x)+\cdots+C_n'(x)\varphi_n(x)=0$, 则
  \[y'=C_1(x)\varphi_1'(x)+\cdots+C_n(x)\varphi_n'(x).\]
  故
  \[y'' = C_1'(x)\varphi_1'(x)+\cdots+C_n'(x)\varphi_n'(x)
    + C_1(x)\varphi_1''(x)+\cdots+C_n(x)\varphi_n''(x).\]
  令 $C_1'(x)\varphi_1'(x)+\cdots+C_n'(x)\varphi_n'(x)=0$, 则
  \[y''=C_1(x)\varphi_1''(x)+\cdots+C_n(x)\varphi_n''(x).\]

  同理可得
  \[y^{(n-1)} = C_1(x)\varphi_1^{(n-1)}(x)+\cdots+C_n(x)\varphi_n^{(n-1)}(x).\]
  \[y^{(n)} = C_1'(x)\varphi_1^{(n-1)}(x)
    + \cdots
    + C_n'(x)\varphi_n^{(n-1)}(x)+C_1(x)\varphi_1^{(n)}(x)+\cdots+C_n(x)\varphi_n^{(n)}(x).\]
  将 $y,y',\cdots,y^{(n)}$ 的表达式代入 $y^{(n)}+a_1(x)y^{(n-1)}+\cdots+a_{n-1}(x)y'+a_n(x)y=f(x)$ 得
  \[\begin{split}
    C_1'(x)\varphi_1^{(n-1)}&(x)+\cdots+C_n'(x)\varphi_n^{(n-1)}(x) \\
    & +\sum_{i=1}^nC_i(x)\Bigl(\varphi_i^{(n)}(x)+a_1(x)\varphi_i^{(n-1)}(x)+\cdots+a_n(x)\varphi_i(x)\Bigr)=f(x).
  \end{split}\]
  而
  \[\sum_{i=1}^nC_i(x)\left(\varphi_i^{(n)}(x)+a_1(x)\varphi_i^{(n-1)}(x)+\cdots+a_n(x)\varphi_i(x)\right)=0.\]
  故
  \[C_1'(x)\varphi_1^{(n-1)}(x)+\cdots+C_n'(x)\varphi_n^{(n-1)}(x)=f(x).\]
  再根据前面所得有
  \[\begin{cases}
    C_1'(x)\varphi_1(x)+\cdots+C_n'(x)\varphi_n(x)=0, \\
    C_1'(x)\varphi_1'(x)+\cdots+C_n'(x)\varphi_n'(x)=0, \\
    \cdots\cdots\\
    C_1'(x)\varphi_1^{(n-1)}(x)+\cdots+C_n'(x)\varphi_n^{(n-1)}(x)=f(x).
  \end{cases}\]
  上述方程组的系数行列式即为 $W(x)$, 故
  \[C_1'(x)
    = \frac{1}{W(x)}
    \begin{vmatrix}
      0&\varphi_2(x)&\cdots&\varphi_n(x)\\
      0&\varphi_2'(x)&\cdots&\varphi_n'(x)\\
      \vdots&\vdots&&\vdots\\
      f(x)&\varphi_2^{(n-1)}(x)&\cdots&\varphi_n^{(n-1)}(x)
    \end{vmatrix}=\frac{W_1(x)}{W(x)}f(x).\]
  同理可得 $\displaystyle C_i'(x)=\frac{W_i(x)}{W(x)}f(x)$ $(i=2,3,\cdots,n)$, 积分得
  \[C_i(x)=\int_{x_0}^x\frac{W_i(s)}{W(s)}f(s)\diff s+C_i.\]
  再代回最初的式子即得
  \[y = C_1\varphi_1(x)+\cdots
    + C_n\varphi_n(x)+\sum_{k=1}^n\varphi_k(x)\int_{x_0}^x\frac{W_k(s)}{W(s)}f(s)\diff s.\qedhere\]
\end{proof}



\begin{exercise}
  设欧拉方程
  \[x^ny^{(n)}+a_1x^{n-1}y^{(n-1)}+\cdots+a_{n-1}xy'+a_ny=0,\]
  其中 $a_1,a_2,\cdots,a_n$ 都是常数, $x>0$. 试利用适当的变换把它化成常系数的齐次线性微分方程.
\end{exercise}

\begin{solution} 
  令 $x=\e^t$, 则
  \begin{align*}
  & \frac{\diff y}{\diff x}=\frac{\diff y}{\diff t}\cdot\frac{\diff t}{\diff x}=\e^{-t}\frac{\diff y}{\diff t}, \\
  & \frac{\diff^2y}{\diff x^2}=\frac{\diff}{\diff t}\left(\frac{\diff y}{\diff x}\right)\cdot\frac{\diff t}{\diff x}=\e^{-2t}\left(\frac{\diff^2y}{\diff t^2}-\frac{\diff y}{\diff t}\right).
  \end{align*}
  用归纳法可以证明
  \begin{equation}
    \frac{\diff^ky}{\diff x^k} = \e^{-kt}\left(\frac{\diff^ky}{\diff t^k}
    + \beta_1\frac{\diff^{k-1}y}{\diff t^{k-1}}
    + \cdots+\beta_{k-1}\frac{\diff y}{\diff t}\right). \tag{$\star$}
  \end{equation}
  其中 $\beta_1,\beta_2,\beta_{k-1}$ 都是常数. 将其代入原方程就得到常系数齐次线性微分方程
  \[\frac{\diff^ny}{\diff t^n}+b_1\frac{\diff^{n-1}y}{\diff t^{n-1}}+\cdots+b_{n-1}\frac{\diff y}{\diff t}+b_ny=0,\]
  其中 $b_1,b_2,\cdots,b_n$ 是常数. 求解之, 再代回原变量, 便可得原方程通解.
\end{solution}

\begin{remark}
  $(\star)$ 式其实不太精细, 事实上, 利用归纳法容易证明下列关系式
  \[x^k\frac{\diff^ky}{\diff x^k}
  = \frac{\diff}{\diff t}\left(\frac{\diff}{\diff t}-1\right)
  \cdots\left(\frac{\diff}{\diff t}-k+1\right)y.\]
\end{remark}


\begin{exercise}
  求解有阻尼的弹簧振动方程
  \[m\frac{\diff^2x}{\diff t^2}+r\frac{\diff x}{\diff t}+kx=0,\]
  其中 $m,r,k$ 都是正的常数. 并就 $\Delta=r^2-4mk$ 大于, 等于和小于零的不同情况, 说明相应解的物理意义.
\end{exercise}

\begin{solution}
  特征方程为
  \[m\lambda^2+r\lambda+k=0.\]
  特征根为
  \[\lambda_1=\frac{-r+\sqrt{r^2-4mk}}{2m},\quad\lambda_2=\frac{-r-\sqrt{r^2-4mk}}{2m}.\]
  \begin{enumerate}[(i)]
  \item $\Delta>0$ 即大阻尼情形, $\lambda_2<\lambda_1<0$, 通解为
  \[x(t)=C_1\e^{\lambda_1t}+C_2\e^{\lambda_2t},\]
  其中 $C_1,C_2$ 为任意常数. 此时 $\lim_{t\to\infty}x(t)=0$, 并且有
  \begin{enumerate}[(a)]
  \item 当常数 $C_1$ 和 $C_2$ 全为零时, 则 $x(t)\equiv 0$, 即弹簧静止;
  \item 当常数 $C_1$ 和 $C_2$ 有且只有一个为零时, 则 $x(t)$ 保持定号, 即弹簧不能振动;
  \item 当常数 $C_1$ 和 $C_2$ 都不为零时, 此时弹簧最多只能经过一次静止点, 亦即
  \[x(t_0)=C_1\e^{\lambda_1t_0}+C_2\e^{\lambda_2t_0}=0\]
  当且仅当 $-1<\frac{C_1}{C_2}<0$ 异号, 
  而且 $t_0=\frac{1}{\lambda_2-\lambda_1}\ln\left(-\frac{C_1}{C_2}\right)$.
  \end{enumerate}
  \item $\Delta<0$ 即小阻尼情形, 此时 $\lambda_1=\alpha+\upi\beta,\lambda_2=\alpha-\upi\beta$, 
  其中 $\alpha=-\frac{r}{2m}<0,\beta=\frac{\sqrt{-\Delta}}{2m}>0$, 通解为
  \[x(t)=\e^{\alpha t}(C_1\cos\beta t+C_2\sin\beta t)=A\e^{\alpha t}\cos(\beta t-\theta_0).\]
  故 $\lim_{t\to\infty}x(t)=0$, 且
  \begin{enumerate}[(a)]
  \item 当 $A=0$ 时, 弹簧静止;
  \item 当 $A>0$ 时, 弹簧振动.
  \end{enumerate}
  \item $\Delta=0$ 即临界阻尼情形, 有两个相等的特征根 $\lambda_1=\lambda_2=-\frac{r}{2m}$, 通解为
  \[x(t)=\e^{-\frac{r}{2m}t}(C_1+C_2t).\]
  此时 $\lim_{t\to\infty}x(t)=0$ 且 $x(t)$ 至多有一个零点, 故弹簧不振动.\qedhere
  \end{enumerate}
\end{solution}



\begin{exercise}
  求解弹簧振子在无阻尼下的强迫振动方程
  \[m\frac{\diff^2x}{\diff t^2}+kx=p\cos\omega t,\]
  其中 $m,k,p$ 和 $\omega$ 都是正的常数. 
  并对外加频率 $\omega\neq\omega_0$ 和 $\omega=\omega_0$ 两种不同的情况, 
  说明解的物理意义, 这里 $\omega_0=\sqrt{\frac{k}{m}}$是弹簧振子的固有频率.
\end{exercise}

\begin{solution}
  特征方程为
  \[m\lambda^2+k=0.\]
  解得特征根为 $\lambda=\pm\sqrt{\frac{k}{m}}\upi=\pm\omega_0\upi$, 故相应齐次线性微分方程的解为
  \[x(t)=C_1\cos\omega_0t+C_2\sin\omega_0t.\]
  当 $\omega\neq\omega_0$ 时, 方程有特解 $x(t)=A\cos\omega t+B\sin\omega t$, 
  代入原方程得 $A=\frac{p}{k-m\omega^2}$, $B=0$, 故原方程通解为
  \[x(t)=C_1\cos\omega_0t+C_2\sin\omega_0t+\frac{p}{k-m\omega^2}\cos\omega t.\]
  当 $\omega=\omega_0$时, 方程有特解 $x(t)=t(A\cos\omega_0t+B\sin\omega_0t)$,
  代入原方程得 $A=0,B=\frac{p}{2m\omega_0}$, 故原方程通解为
  \[x(t)=C_1\cos\omega_0t+C_2\sin\omega_0t+\frac{p}{2m\omega_0}t\sin\omega_0 t.\]
  此时发生了共振.
\end{solution}



\begin{exercise}
  求解下列常系数线性微分方程:
  \begin{enumerate}[(1)]
  \item $y''+y'-2y=2x,y(0)=0,y'(0)=1$;
  \item $2y''-4y'-6y=3\e^{2x}$;
  \item $y''+2y'=3+4\sin 2x$;
  \item $y'''+3y'-4y=0$;
  \item $y'''-2y''-3y'+10y=0$;
  \item $y'''-3ay''+3a^2y'-a^3y=0$;
  \item $y^{(4)}-4y'''+8y''-8y'+3y=0$;
  \item $y^{(5)}+2y'''+y'=0$;
  \item $y^{(4)}+2y''+y=\sin x,y(0)=1,y'(0)=-2,y''(0)=3,y'''(0)=0$;
  \item $y^{(4)}+y=2\e^x,y(0)=y'(0)=y''(0)=y'''(0)=1$;
  \item $y''-2y'+2y=4\e^x\cos x$;
  \item $y''-5y'+6y=(12x-7)\e^{-x}$;
  \item $x^2y''+5xy'+13y=0(x>0)$;
  \item $(2x+1)^2y''-4(2x+1)y'+8y=0$.
  \end{enumerate}
\end{exercise}

\begin{solution}
  (1) $\lambda^2+\lambda-2=0\Rightarrow\lambda_1=-2,\lambda_2=1$, 
  设方程的特解为 $y=ax+b$, 代入原方程得 $a=-1,b=-\frac{1}{2}$, 故原方程的通解为
  \[y=C_1\e^{-2x}+C_2\e^x-x-\frac{1}{2}.\]
  代入初值条件得 $C_1=-\frac{1}{2},C_2=1$, 故原方程的解为
  \[y=-\frac{1}{2}\e^{-2x}+\e^x-x-\frac{1}{2}.\]

  (2) $2\lambda^2-4\lambda-6=0\Rightarrow\lambda_1=3,\lambda_2=-1$, 
  设方程的特解为 $y=a\e^{2x}$, 代入原方程得 $a=-\frac{1}{2}$, 故原方程的通解为
  \[y=C_1\e^{3x}+C_2\e^{-x}-\frac{1}{2}\e^{2x}.\]

  (3) $\lambda^2+2\lambda=0\Rightarrow\lambda_1=0,\lambda_2=-2$, 
  设特解为 $y=Ax+B\cos 2x+C\sin2x$, 代入原方程得 $A=\frac{3}{2},B=C=-\frac{1}{2}$, 故原方程通解为
  \[y=C_1+C_2\e^{-2c}+\frac{3}{2}x-\frac{1}{2}(\sin2x+\cos2x).\]

  (4) $\lambda^3+3\lambda-4=0\Rightarrow\lambda=1,\frac{-1\pm\sqrt{15}\upi}{2}$, 
  故方程的实基本解组为
  $\e^x$, $\e^{-\frac{1}{2}x}\cos\frac{\sqrt{15}}{2}x$, 
  $\e^{-\frac{1}{2}x}\sin\frac{\sqrt{15}}{2}x$, 故通解为
  \[y=C_1\e^x+\left(C_2\cos\frac{\sqrt{15}}{2}x+C_3\sin\frac{\sqrt{15}}{2}x\right)\e^{-\frac{1}{2}x}.\]

  (5) $\lambda^3-2\lambda^2-3\lambda+10=0\Rightarrow\lambda=-2,2\pm\upi$, 
  故实基本解组为 $\e^{-2x}$, $\e^{2x}\cos x$, $\e^{2x}\sin x$, 故通解为
  \[y=C_1\e^{-2x}+(C_2\cos x+C_3\sin x)\e^{2x}.\]

  (6) $\lambda^3-3a\lambda^2+3a^2\lambda-a^3=(\lambda-a)^3=0\Rightarrow\lambda_{1,2,3}=a$, 故通解为
  \[y=\left(C_1+C_2x+C_3x^2\right)\e^{ax}.\]

  (7) $\lambda^4-4\lambda^3+8\lambda^2-8\lambda+3=(\lambda-1)^2(\lambda^2-2\lambda+3)=0\Rightarrow\lambda_{1,2}=1,\lambda_3=1+\sqrt{2}\upi,\lambda_4=1-\sqrt{2}\upi$, 故通解为
  \[y=(C_1+C_2x)\e^x+(C_3\cos\sqrt{2}x+C_4\sin\sqrt{2}x)\e^x.\]

  (8) $\lambda^5+2\lambda^3+\lambda=\lambda(\lambda^2+1)^2=0\Rightarrow\lambda_1=0,\lambda_{2,3}=\upi,\lambda_{4,5}=-\upi$, 
  故复基本解组为 $1$, $\e^{\upi x}$, $x\e^{\upi x}$, $\e^{-\upi x}$, $x\e^{-\upi x}$, 
  相应的实基本解组为 $1$, $\cos x$, $\sin x$, $x\cos x$, $x\sin x$, 故通解为
  \[y=C_1+(C_2+C_3x)\cos x+(C_4+C_5x)\sin x.\]

  (9) $\lambda^4+2\lambda^2+1=(\lambda^2+1)^2=0\Rightarrow\lambda_{1,2}=\upi,\lambda_{3,4}=-\upi$,
  故对应齐次方程的通解为
  \[\varphi(x)=C_1\sin x+C_2\cos x+C_3x\sin x+C_4x\cos x.\]
  设特解为 $\varphi^*(x)=x^2(A\cos x+B\sin x)$, 则
  \[\begin{split}
    \left(\varphi^*(x)\right)'&=2x(A\cos x+B\sin x)+x^2(-A\sin x+B\cos x), \\
    \left(\varphi^*(x)\right)''&=(2-x^2)(A\cos x+B\sin x)+4x(-A\sin x+B\cos x), \\
    \left(\varphi^*(x)\right)'''&=-6x(A\cos x+B\sin x)+(6-x^2)(-A\sin x+B\cos x), \\
    \left(\varphi^*(x)\right)^{(4)}&=(x^2-12)(A\cos x+B\sin x)-8x(-A\sin x+B\cos x).
  \end{split}\]
  故 $(x^2-12+4-2x^2+x^2)(A\cos x+B\sin x)+(-8x+8x)(-A\sin x+B\cos x)=-8(A\cos x+B\sin x)=\sin x\Rightarrow A=0,B=-\frac{1}{8}$, 
  故 $\varphi^*(x)=-\frac{1}{8}x^2\sin x$, 故原方程的通解为
  \[y=C_1\sin x+C_2\cos x+C_3x\sin x+C_4x\cos x-\frac{1}{8}x^2\sin x.\]
  再代入初值条件 $y(0)=C_2=1$, $y'(0)=C_1+C_4=-2$, $y''(0)=-C_2+2C_3=3$, 
  $y'''(0)=-C_1-3C_4-\frac{3}{4}=0$, 
  解得 $C_1=-\frac{21}{8},C_2=1,C_3=2,C_4=\frac{5}{8}$, 故原方程的解为
  \[y=\left(-\frac{1}{8}x^2+2x-\frac{21}{8}\right)\sin x+\left(\frac{5}{8}x+1\right)\cos x.\]

  (10) $\lambda^4+1=0\Rightarrow\lambda=\e^{\upi\left(\frac{\pi}{4}+\frac{1}{2}k\pi\right)}$ 
  $(k=0,1,2,3)$, 
  也即 
  $\lambda_1=\frac{\sqrt{2}}{2}+\frac{\sqrt{2}}{2}\upi$,
  $\lambda_2=-\frac{\sqrt{2}}{2}+\frac{\sqrt{2}}{2}\upi$,
  $\lambda_3=-\frac{\sqrt{2}}{2}-\frac{\sqrt{2}}{2}\upi$,
  $\lambda_4=\frac{\sqrt{2}}{2}-\frac{\sqrt{2}}{2}\upi$, 
  故相应的齐次线性微分方程的解为
  \[\varphi(x) = 
  \left(C_1\cos\frac{\sqrt{2}}{2}x+C_2\sin\frac{\sqrt{2}}{2}x\right)\e^{\frac{\sqrt{2}}{2}x}
  + \left(C_3\cos\frac{\sqrt{2}}{2}x+C_4\sin\frac{\sqrt{2}}{2}x\right)\e^{-\frac{\sqrt{2}}{2}x}.\]
  设原方程的特解为 $\varphi^*(x)=A\e^x$, 代入原方程得 $A=1$, 故特解为 $\varphi^*(x)=\e^x$, 因此原方程的通解为
  \[y = \left(C_1\cos\frac{\sqrt{2}}{2}x+C_2\sin\frac{\sqrt{2}}{2}x\right)\e^{\frac{\sqrt{2}}{2}x}
  + \left(C_3\cos\frac{\sqrt{2}}{2}x+C_4\sin\frac{\sqrt{2}}{2}x\right)\e^{-\frac{\sqrt{2}}{2}x}+\e^x.\]
  结合初值条件知 $C_i=0$ $(i=1,2,3,4)$, 故原方程满足初值条件的解为 $y=\e^x$.

  (11) $\lambda^2-2\lambda+2=0\Rightarrow\lambda=1\pm\upi$, 故相应齐次线性微分方程的通解为
  \[\varphi(x)=(C_1\cos x+C_2\sin x)\e^x.\]
  设原方程的特解为 $\varphi^*(x)=x(A\cos x+B\sin x)\e^x$, 代入原方程得 $A=0,B=2$, 
  故特解为 $\varphi^*(x)=2x\e^x\sin x$, 故原方程的通解为
  \[y=(C_1\cos x+C_2\sin x)\e^x+2x\e^x\sin x.\]

  (12) $\lambda^2-5\lambda+6=0\Rightarrow\lambda_1=2,\lambda_2=3$, 故相应齐次线性微分方程的通解为
  \[\varphi(x)=C_1\e^{2x}+C_2\e^{3x}.\]
  设原方程的特解为 $\varphi^*(x)=(Ax+B)\e^{-x}$, 代入原方程得 $A=1,B=0$, 
  故特解为 $\varphi^*(x)=x\e^{-x}$, 因此原方程的通解为
  \[y=C_1\e^{2x}+C_2\e^{3x}+x\e^{-x}.\]

  (13) 令 $x=\e^t$, 则原方程化为
  \[\frac{\diff}{\diff t}\left(\frac{\diff}{\diff t}-1\right)y+5\frac{\diff}{\diff t}y+13y
    = \frac{\diff^2y}{\diff t^2}+4\frac{\diff y}{\diff t}+13y=0.\]
  特征方程为 $\lambda^2+4\lambda+13=0\Rightarrow\lambda=-2\pm3\upi$, 
  故实基本解组为 $\e^{-2t}\cos3t,\e^{-2t}\sin3t$, 
  代回原变量即得基本解组为 $\frac{1}{x^2}\cos(3\ln x),\frac{1}{x^2}\sin(3\ln x)$, 故通解为
  \[y=\frac{1}{x^2}(C_1\cos(3\ln x)+C_2\sin(3\ln x)).\]

  (14) 令$u=2x+1$, 则
  \[\frac{\diff y}{\diff x}=\frac{\diff y}{\diff u}\frac{\diff u}{\diff x}=2\frac{\diff y}{\diff u},\]
  \[\frac{\diff^2y}{\diff x^2}=\frac{\diff}{\diff u}\left(2\frac{\diff y}{\diff u}\right)\frac{\diff u}{\diff x}=4\frac{\diff^2y}{\diff u^2}.\]
  故原方程化为
  \[u^2\cdot4\frac{\diff^2y}{\diff u^2}-4u\cdot2\frac{\diff y}{\diff u}+8y=0
    \Rightarrow u^2\frac{\diff^2y}{\diff u^2}-2u\frac{\diff y}{\diff u}+2y=0.\]
  令 $u=\e^t$, 则上述方程化为
  \[\frac{\diff}{\diff t}\left(\frac{\diff}{\diff t}-1\right)y-2\frac{\diff}{\diff t}y+2y
    = \frac{\diff^2y}{\diff t^2}-3\frac{\diff y}{\diff t}+2y=0.\]
  特征方程为 $\lambda^2-3\lambda+2=0\Rightarrow\lambda_1=1,\lambda_2=2$, 故通解为
  \[y=C_1\e^t+C_2\e^{2t}=C_1u+C_2u^2=C_1(2x+1)+C_2(2x+1)^2.\qedhere\]
\end{solution}