\chapter{奇解}



\section{一阶隐式微分方程}



\subsection{证明与总结}



考虑一阶隐式微分方程:
\begin{equation}
  F\left(x,y,\frac{\diff y}{\diff x}\right)=0.  \tag{$\star$}
\end{equation}

\textbullet 若可由 $(\star)$ 式导出 $y=f(x,p)$, 则
\[p=f'_x(x,p)+f'_p(x,p)\frac{\diff y}{\diff x}.\]

若解出 $p=u(x,C)$, 则通解为 $y=f(x,u(x,C))$;
若解出 $x=v(p,C)$, 则通解为
\[\begin{cases}
  x=v(p,C), \\ y=f(v(p,C),p),
\end{cases}\]
其中 $p$ 视作参变量.

\textbullet 若可由 $(\star)$ 式导出 $F(y,p)=0$, 即不显含自变量 $x$, 
设 $y=g(t),p=h(t)$, 则
\[\diff x=\frac{1}{p}\diff y=\frac{g'(t)}{h(t)}\diff t\Rightarrow x=\int\frac{g'(t)}{h(t)}\diff t.\]
故通解为 $x=\int\frac{g'(t)}{h(t)}\diff t$, $y=g(t)$.

\textbullet 若可由 $(\star)$ 式导出 $F(x,p)=0$, 即不显含未知函数 $y$, 此时处理方式同上.

\textbullet 设 $x=f(u,v),y=g(u,v),p=h(u,v)$, 由 $\diff y=p\diff x$ 若能解出 $v=Q(u,C)$, 
则通解为 $x=f(u,Q(u,C)),y=g(u,Q(u,C))$.



\subsection{习题}



\begin{exercise}
  求解下列微分方程:
  \begin{enumerate}[(1)]
  \item $\displaystyle 2y=p^2+4px+2x^2\left(p=\frac{\diff y}{\diff x}\right)$;
  \item $\displaystyle y=px\ln x+(xp)^2$;
  \item $\displaystyle 2xp=2\tan y+p^3\cos^2y$.
  \end{enumerate}
\end{exercise}

\begin{proof}
  (1) 将原方程求导得
  \[2p=2p\frac{\diff p}{\diff x}+4p+4x\frac{\diff p}{\diff x}+4x.\]
  即
  \[(p+2x)\biggl(\frac{\diff p}{\diff x}+1\biggr)=0.\]

  当 $p+2x=0$ 时, 代入原方程得特解 $y=-x^2$;

  当 $\frac{\diff p}{\diff x}=-1$ 时, $p=-x+C$,
  代入原方程得通解 $y=-\frac{1}{2}x^2+Cx+\frac{1}{2}C^2$.

  (2) 将原方程求导得
  \[p=x\ln x\frac{\diff p}{\diff x}+p(\ln x+1)+2xp\left(p+x\frac{\diff p}{\diff x}\right).\]
  即
  \[\left(x\frac{\diff p}{\diff x}+p\right)(\ln x+2xp)=0.\]

  当 $x\frac{\diff p}{\diff x}+p=0$ 时, $p=\frac{C}{x}$, 代入原方程得通解 $y=C\ln x+C^2$;

  当 $\ln x+2xp=0$ 时, $p=-\frac{\ln x}{2x}$, 代入原方程得特解 $y=-\frac{1}{4}(\ln x)^2$.

  (3) 当 $p=0$ 时得特解 $y=k\pi$ ($k\in\mathbb{Z}$);

  当 $p\neq 0$ 时, 将原方程变形为
  \[x=\frac{\tan y}{p}+\frac{1}{2}p^2\cos^2y.\]
  在上式中关于 $y$ 求导得
  \[\frac{1}{p}=\frac{1}{p^2}\left(\sec^2y\cdot p-\tan y\cdot\frac{\diff p}{\diff y}\right)+p\frac{\diff p}{\diff y}\cos^2y-p^2\sin y\cos y.\]
  即
  \[\left(\frac{\diff p}{\diff y}-p\tan y\right)\left(p^3\cos^2y-\tan y\right)=0.\]

  当 $\frac{\diff p}{\diff y}-p\tan y=0$ 时, 
  解得 $p=\frac{1}{C\cos y}$ $(C\neq 0)$, 代入原方程得通解 $x=C\sin y+\frac{1}{2C^2}$;

  当 $p^3\cos^2y-\tan y=0$ 时, $p=\frac{\tan^{\frac{1}{3}}y}{\cos^{\frac{2}{3}}y}$,
  代入原方程得特解 $x=\frac{3}{2}\sin^{\frac{2}{3}}y$.
\end{proof}



\begin{exercise}
  用参数法求解下列微分方程:
  \begin{enumerate}[(1)]
  \item $\displaystyle 2y^2+5\left(\frac{\diff y}{\diff x}\right)^2=4$;
  \item $\displaystyle x^2-3\left(\frac{\diff y}{\diff x}\right)^2=1$;
  \item $\displaystyle\left(\frac{\diff y}{\diff x}\right)^2+y-x^2=0$;
  \item $\displaystyle x^3+\left(\frac{\diff y}{\diff x}\right)^3=4x\frac{\diff y}{\diff x}$.
  \end{enumerate}
\end{exercise}

\begin{proof} 
  (1) 令 $y=\sqrt{2}\sin t,\frac{\diff y}{\diff x}=\frac{2}{\sqrt{5}}\cos t$, 则

  当 $\diff y=0$ 时, 得特解 $y=\pm\sqrt{2}$;

  当 $\diff y\neq 0$时, $\diff x=\frac{\diff x}{\diff y}\diff y=\frac{\sqrt{5}}{2\cos t}\sqrt{2}\cos t\diff t=\sqrt{\frac{5}{2}}\diff t\Rightarrow x=\sqrt{\frac{5}{2}}t+C$, 
  故通解为 $y=\sqrt{2}\sin\left(\sqrt{\frac{2}{5}}(x-C)\right)$.

  (2) 令 $x=\sec t,\frac{\diff y}{\diff x}=\frac{1}{\sqrt{3}}\tan t$, 则
  \[\diff y=\frac{\diff y}{\diff x}\diff x=\frac{1}{\sqrt{3}}\tan^2t\sec t\diff t.\]
  积分得
  \[y=\frac{1}{2\sqrt{3}}(\sec t\tan t-\ln|\sec t+\tan t|)+C.\]
  故通解为
  \[\begin{cases}
    x = \sec t, \\
    y = \frac{1}{2\sqrt{3}}(\sec t\tan t-\ln|\sec t+\tan t|)+C.
  \end{cases}\]
  
  本题也可以利用恒等式: $\cosh^2t-\sinh^2t=1$, 
  得到另外一种通解表达式: $x=\cosh t,y=\frac{1}{4\sqrt{3}}(\sinh2t-2t)+C$.

  (3) 将原方程求导得
  \[2p\frac{\diff p}{\diff x}+p-2x=0\Rightarrow \frac{\diff p}{\diff x}=\frac{x}{p}-\frac{1}{2}.\] 
  令 $u=\frac{p}{x}\neq 0$, 则
  \[\frac{\diff p}{\diff x}=u+x\frac{\diff u}{\diff x}=\frac{1}{u}-\frac{1}{2}\Rightarrow x\frac{\diff u}{\diff x}+\frac{2u^2+u-2}{2u}=0.\]

  \textbullet 当 $2u^2+u-2=0$ 即 $u=\frac{-1\pm\sqrt{17}}{4}$ 时,
  若 $u=\frac{-1+\sqrt{17}}{4}$, 
  则 $p=\frac{\diff y}{\diff x}=\frac{-1+\sqrt{17}}{4}x\Rightarrow y=\frac{-1+\sqrt{17}}{8}x^2+C$, 
  将 $p=\frac{-1+\sqrt{17}}{4}x$ 直接代入原方程得 $y=\frac{-1+\sqrt{17}}{8}x^2$, 
  故原方程有特解 $y=\frac{-1+\sqrt{17}}{8}x^2$, 同理可得到另外一个特解 $y=\frac{-1-\sqrt{17}}{8}x^2$;

  \textbullet 当 $2u^2+u-2\neq 0$ 时, $\frac{2u}{2u^2+u-2}\diff u+\frac{1}{x}\diff x=0$,
  下面对该变量分离的方程进行不定积分:
  \begin{align*}
    & \int\frac{2u}{2u^2+u-2}\diff u+\int\frac{1}{x}\diff x=0 \\
    \Longrightarrow{} & \frac{1}{2}\ln|2u^2+u-2|-\frac{1}{2}\int\frac{1}{2\left(x+\frac{1}{4}\right)^2-\frac{17}{8}}\diff u+\ln|x|=C \\
    \Longrightarrow{} & \frac{1}{2}\ln|2u^2+u-2|-\frac{1}{4}\int\frac{1}{\left(x+\frac{1}{4}\right)^2-\frac{17}{16}}\diff u+\ln|x|=C \\
    \Longrightarrow{} & \frac{1}{2}\ln|2u^2+u-2|+\frac{1}{2\sqrt{17}}\ln\left|\frac{u+\frac{1+\sqrt{17}}{4}}{u+\frac{1-\sqrt{17}}{4}}\right|+\ln|x|=C \\
    \Longrightarrow{} & \frac{\sqrt{17}}{4}\ln|2p^2-2x^2+px|+\frac{1}{4}\ln\left|\frac{p+\frac{1+\sqrt{17}}{4}x}{p+\frac{1-\sqrt{17}}{4}x}\right|=C_1 \\
    \Longrightarrow{} & (2(p-\alpha x)(p-\beta x))^{\frac{\sqrt{17}}{4}}\left(\frac{p-\beta x}{p-\alpha x}\right)^{\frac{1}{4}}=C_2\neq 0 \\
    \Longrightarrow{} & (p-\alpha x)^{\alpha}=C_3(p-\beta x)^{\beta},C_3\neq 0,
  \end{align*}
  其中 $\alpha=\frac{\sqrt{17}-1}{4}$, $\beta=\frac{-\sqrt{17}-1}{4}$.

  综上所述, 原方程有通解 $(p-\alpha x)^{\alpha}=C(p-\beta x)^{\beta}$ $(C\neq 0)$
  以及两个特解 $y_1=\frac{1}{2}\alpha x^2,y_2=\frac{1}{2}\beta x^2$.

  (4) 令 $p=tx$, 则 $x^3(1+t^3)=4tx^2\Rightarrow x=\frac{4t}{1+t^3},p=\frac{4t^2}{1+t^3}$, 故
  \[\diff y=p\diff x=\frac{4t^2}{1+t^3}\frac{4(1+t^3)-4t\cdot3t^2}{(1+t^3)^2}\diff t
    = \frac{4t^2(4-8t^2)}{(1+t^3)^3}\diff t.\]
  积分得
  \[y=\int\frac{4t^2(4-8t^2)}{(1+t^3)^3}\diff t=\frac{32}{3}\frac{1}{1+t^3}-\frac{8}{(1+t^3)^2}+C.\]
  故通解为
  \[\begin{cases}
    x=\frac{4t}{1+t^3}, \\
    y=\frac{32}{3}\frac{1}{1+t^3}-\frac{8}{(1+t^3)^2}+C.
  \end{cases}\qedhere\]
\end{proof}



\section{奇解}



\begin{exercise}
  利用 $p$-判别式求下列微分方程的奇解:
  \begin{enumerate}[(1)]
  \item $\displaystyle y=x\frac{\diff y}{\diff x}+\left(\frac{\diff y}{\diff x}\right)^2$;
  \item $\displaystyle y=2x\frac{\diff y}{\diff x}+\left(\frac{\diff y}{\diff x}\right)^2$;
  \item $\displaystyle (y-1)^2\left(\frac{\diff y}{\diff x}\right)^2=\frac{4}{9}y$;
  \end{enumerate}
\end{exercise}

\begin{solve} 
  (1) 令 $F(x,y,p)=y-xp-p^2$, 则 $p$-判别式为
  $y-xp-p^2=0,-x-2p=0\Rightarrow y=-\frac{1}{4}x^2$, 经验证 $y=-\frac{1}{4}x^2$ 是原方程的解, 又
  \[F_y'\left(x,-\frac{1}{4}x^2,-\frac{1}{2}x\right)=1,\:
    F_{pp}''\left(x,-\frac{1}{4}x^2,-\frac{1}{2}x\right)=-2,\:
    F_p'\left(x,-\frac{1}{4}x^2,-\frac{1}{2}x\right)=0.\]
  故 $y=-\frac{1}{4}x^2$ 是奇解.

  (2) 令 $F(x,y,p)=y-2xp-p^2$, 则$p$-判别式为$y-2xp-p^2=0,-2x-2p=0\Rightarrow y=-x^2$,
  但是 $y=-x^2$ 不是原方程的解更不是奇解.

  (3) 令 $F(x,y,p)=(y-1)^2p^2-\frac{4}{9}y$, 
  则 $p$-判别式为 $(y-1)^2p^2-\frac{4}{9}y=0,2(y-1)^2p=0\Rightarrow y=0$, 
  经验证 $y=0$ 是原方程的解, 又
  \[F_y'(x,0,0)=-\frac{4}{9},\: F_{pp}''(x,0,0)=2,\: F_p'(x,0,0)=0,\]
  故 $y=0$ 是原方程的奇解.
\end{solve}



\begin{exercise}
  举例说明, 在定理 4.2 的条件 (4.28) 中的两个不等式是缺一不可的.
\end{exercise}

\begin{solve}  
  分别考虑方程 $\displaystyle\left(\frac{\diff y}{\diff x}\right)^2-y^2=0$
  与 $\displaystyle\sin\left(y\frac{\diff y}{\diff x}\right)=y$.
\end{solve}



\begin{exercise}
  研究下面的例子, 说明定理 4.2 的条件 (4.29) 是不可缺少的:
  \[y=2x+y'-\frac{1}{3}(y')^3.\]
\end{exercise}

\begin{solve} 
  $p$-判别式为: $y=2x+p-\frac{1}{3}p^3$, $0 = 1-p^2$, 解得 $y = 2x\pm\frac{2}{3}$, 
  经检验 $y=2x+\frac{2}{3}$ 不是原方程的解. 而 $y=2x-\frac{2}{3}$ 是原方程的解, 但不是奇解.

  下面证明 $y = 2x - \frac23$ 不是奇解.

  令 $F(x,y,p) = y-2x+\frac{1}{3}p^3-p$, 则
  \[F_y'\left(x,2x-\frac{2}{3},2\right)=1,\:
    F_{pp}''\left(x,2x-\frac{2}{3},2\right)=4,\:
    F_p'\left(x,2x-\frac{2}{3},2\right)=3\neq 0.\]
  故条件 $F_p'(x,\varphi(x),\varphi'(x))=0$ 不可缺少.
\end{solve}



\section{包络}



\begin{exercise}
  试求克莱罗方程的通解及其包络.
\end{exercise}

\begin{solve} 
  克莱罗方程为: $y=xp+f(p)\left(p=\frac{\diff y}{\diff x}\right)$, 其中 $f''(p)\neq0$.
  \[y=xp+f(p)\Rightarrow(x+f'(p))\frac{\diff p}{\diff x}=0.\]
  由 $\frac{\diff p}{\diff x}=0 $ 即 $p=C$ 得通解 $y=Cx+f(C)$, $C$ 判别式为
  \[\begin{cases}
      y=Cx+f(C), \\
      x+f'(C)=0
    \end{cases}\Rightarrow
    \begin{cases}
      x=-f'(C)=\varphi(C), \\
      y=-Cf'(C)+f(C)=\psi(C).
    \end{cases}(*)\]
  令 $V(x,y,C)=Cx+f(C)-y$, 则 $V_x'(\varphi(C),\psi(C),C)=C,V_y'(\varphi(C),\psi(C),C)=-1$, 
  故 $(V_x',V_y')\neq(0,0)$, 又$(\varphi'(C),\psi'(C))=(-f''(C),-Cf''(C))\neq(0,0)$, 
  故 $(*)$ 是曲线族 $y=Cx+f(C)$ 的一支包络.
\end{solve}



\begin{exercise}
  试求一微分方程, 使它有奇解为$y=\sin x$.
\end{exercise}

\begin{solve} 
  考虑克莱罗方程 $y=xp+f(p)$, 将 $y=\sin x$ 代入得
  \[\sin x=x\cos x+f(\cos x).\]
  令 $\cos x=p$, 得
  \[\sqrt{1-p^2}=p\arccos p+f(p).\]
  故
  \[f(p)=-p\arccos p+\sqrt{1-p^2}.\]
  容易验证 $y=\sin x$ 是方程 $y=xp-p\arccos p+\sqrt{1-p^2}$ 的奇解.
\end{solve}