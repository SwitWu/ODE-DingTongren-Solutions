\chapter{边值问题}

\section{施图姆比较定理}

\begin{exercise}
  如果在定理 2 中假设
  \[R(x) > Q(x)\quad (x\in J),\]
  则定理的结论可以加强到: $x_1 < x_0 < x_2$.
\end{exercise}


\begin{exercise}
  如果微分方程
  \[y'' + Q(x)y = 0\]
  中的系数函数 $Q(x)$ 满足不等式
  \[Q(x) \leq M\quad (a\leq x<\infty),\]
  其中常数 $M>0$. 则它的任何非零解 $y = \varphi(x)$ 的相邻零点的间距不小于 $\pi/\sqrt{M}$.
\end{exercise}


\begin{exercise}
  利用定理 9.2 证明: 贝塞尔函数 $\mathrm{J}_n(x)$ 和诺伊曼函数 $\mathrm{Y}_n(x)$
  都有无穷多个零点, 而且它们各自相邻零点的间距当 $x\to\infty$ 时趋于 $\pi$.
\end{exercise}


\begin{exercise}
  设微分方程
  \[\frac{\diff^2 x}{\diff t^2} + P(t)x = 0,\]
  其中 $P(t)$ 是 $t$ 的连续函数, 而且满足
  \[n^2 < P(t) < (n+1)^2,\]
  这里 $n$ 是一个非负的整数.
  则上述方程的任何非零解都不是 $2\pi$ 周期的.
\end{exercise}


\begin{exercise}
  利用定理 9.2 证明: 齐次线性微分方程 (9.1) 的任何两个线性无关的解的零点是互相交错的.
\end{exercise}