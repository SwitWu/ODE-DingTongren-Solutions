\chapter{定性理论与分支理论初步}



\setcounter{section}{1}
\section{解的李雅普诺夫稳定性}

定理 8.2, 8.3 和 8.4 的证明见 \cite{R-Yuan}.

\begin{exercise}
  证明: 线性方程零解的渐进稳定性等价于它的全局稳定性.
\end{exercise}

\begin{proof}
  只需要证明渐进稳定性蕴含全局渐进稳定性.
  设 $X(t)$ 为基解矩阵, 则过 $(t_0, x_0)$ 的解可以表为
  \[x(t; t_0,x_0) = X(t) X^{-1}(t_0)x_0.\]
  因为零解吸引, 所以存在 $\delta_0>0$, 使得当 $|\tilde{x}_0| < \delta_0$ 时, 有
  \[x(t; x_0, \tilde{x}_0) \to 0,\quad t\to\infty.\]
  对任意 $x_0$, 存在常数 $c\neq 0$ 使得 $|c x_0| < \delta_0$, 故
  \[x(t; t_0, cx_0) = X(t)X^{-1}(x_0)(cx_0) = c X(t)X^{-1}(t_0)x_0 \to 0,\quad t\to\infty.\]
  于是 $X(t)X^{-1}(t_0)x_0\to 0$ ($t\to\infty$), 这说明零解是全局渐进稳定的.
\end{proof}


\begin{exercise}
  设 $x$ 和 $t$ 都是标量, 试求出方程
  \[\frac{\diff x}{\diff t} = a(t)x\]
  的零解为稳定或渐进稳定的充要条件.
\end{exercise}

\begin{solution}
  方程的解为
  \[x(t) = C \e^{\int_0^t a(s) \diff s}.\]
  于是零解稳定的充要条件为
  \[\int_0^{\infty} a(s) \diff s < \infty.\]
  零解渐进稳定的充要条件为
  \[\int_0^{\infty} a(s) \diff s = -\infty. \qedhere\]
\end{solution}


\begin{exercise}
  对于极坐标下的方程
  \[\dot{\theta} = 1,\quad \dot{\theta} = \begin{cases}
    r^2\sin\frac{1}{r}, & \text{当\ } r>0, \\
    0, & \text{当\ } r=0,
  \end{cases}\]
  试作出原点附近的相图, 并研究平衡点 $r=0$ 的稳定性质.
\end{exercise}



\begin{exercise}
  设二阶常系数线性方程
  \[\frac{\diff\bm{x}}{\diff t} = \bm{A}\bm{x},\]
  其中 $\bm{A}$ 是一个 $2\times 2$ 的常矩阵. 记
  \[\begin{cases}
    p = -\tr\bm{A}, \\
    q = \det\bm{A}.
  \end{cases}\]
  再设 $p^2+q^2\neq 0$, 试证:
  \begin{enumerate}[(1)]
    \item 当 $p>0$ 且 $q>0$ 时, 零解是渐进稳定的;
    \item 当 $p>0$ 且 $q=0$ 或 $p=0$ 且 $q>0$ 时, 零解是稳定的, 但不是渐进稳定的;
    \item 在其他情形下, 零解都是不稳定的.
  \end{enumerate}
\end{exercise}

\begin{proof}
  特征多项式为 $\chi_A(\lambda) = \lambda^2 + p\lambda +q$. 于是特征根 $\lambda_1$,
  $\lambda_2$ 满足
  \[\lambda_1 + \lambda_2 = -p,\qquad \lambda_1\lambda_2 = q.\]

  (1) 当 $p>0$ 且 $q>0$ 时, $\lambda_1$ 和 $\lambda_2$ 的实部都为负, 因此零解是渐进稳定的.

  (2) 当 $p>0$ 且 $q=0$ 或 $p=0$ 且 $q>0$ 时, $\lambda_1$ 和 $\lambda_2$ 一个为零, 一个为负,
  因此零解是稳定的但不是渐进稳定的.

  (3) 其它情形下, 零解不是稳定的.
\end{proof}



\begin{exercise}
  讨论二维的微分方程
  \[\dot{x} = y-xf(x,y),\quad \dot{y} = -x-yf(x,y)\]
  零解的稳定性, 其中函数 $f(x,y)$ 在 $(0,0)$ 附近是连续可微的.
\end{exercise}

\begin{solution}
  取 $V(x,y) = \frac12 (x^2+y^2)$, 则
  \[\frac{\diff V}{\diff t}(x(t), y(t)) = -(x^2+y^2) f(x,y).\]
  由定理 8.4 知, 若存在 $\varepsilon>0$, 使得当 $0< x^2 + y^2 < \varepsilon$ 时,
  $f(x,y) > 0$ ($f(x,y)\geq 0$, 或 $f(x,y)<0$),
  则零解是渐进稳定的 (稳定的, 或不稳定的).
\end{solution}



\begin{exercise}
  设 $x\in\mathbb{R}^1$, 函数 $g(x)$ 连续, 且 $xg(x)>0$ 当 $x\neq 0$. 试证方程
  \[\ddot{x}+g(x) = 0\]
  的零解是稳定的, 但不是渐进稳定的.
\end{exercise}

\begin{proof}
  令 $y = \dot{x}$, 则 $\ddot{x} + g(x) = 0$ 等价于
  \[\begin{cases}
    \dot{x} = y, \\
    \dot{y} = -g(x).
  \end{cases}\]
  取 $V(x,y) = \frac{1}{2}y^2 + \int_0^x g(s) \diff s$, 则由条件可知
  $V(x,y)$ 满足\textbf{条件 I}, 又因为
  \[\frac{\diff V}{\diff t} = 0,\]
  因此零解是稳定的, 但不是渐进稳定的.
\end{proof}



\begin{exercise}
  研究二维微分方程
  \[\dot{x} = y,\quad \dot{y} = -1+x^2\]
  的两个平衡点的稳定性.
\end{exercise}

\begin{solution}
  方程有两个平衡点 $(1,0)$ 和 $(-1,0)$.

  (1) 考虑 $(1,0)$ 点. 令 $u = x-1$, $v = y$, 则方程变为
  \[\frac{\diff u}{\diff t} = v,\qquad \frac{\diff v}{\diff t} = u^2+2u.\]
  线性部分的特征函数为 $\lambda^2-2=0$, 特征根是 $\lambda_{1,2} = \pm\sqrt{2}$,
  它有一个正实部的根, 所以 $(1,0)$ 是不稳定的.

  (2) 考虑 $(-1,0)$ 点. 令 $u = x+1$, $v=y$, 则方程变为
  \[\frac{\diff u}{\diff t} = v,\qquad \frac{\diff v}{\diff t} = -2u + u^2.\]
  线性部分的特征根为纯虚根 $\pm\sqrt{2}\mathrm{i}$.
  取 $V = u^2 + \frac{1}{2}v^2 -\frac{1}{2}u^3$, 则 $V$ 在 $(0,0)$
  附近是定正的, 且
  \[\frac{\diff V}{\diff t} = (2u-u^2) \dot{u} + v\dot{v} = 0,\]
  因此 $(-1,0)$ 是稳定的.
\end{solution}



\begin{exercise}
  讨论下列微分方程零解的稳定性:
  \begin{enumerate}[(1)]
    \item $\dot{x} = -y-xy^2$, $\dot{y} = x-x^4y$;
    \item $\dot{x} = -y^3-x^5$, $\dot{y} = x^3-y^5$;
    \item $\dot{x} = -x+2x(x+y)^2$, $\dot{y} = -y^3+2y^3(x+y)^2$;
    \item $\dot{x} = 2x^2y+y^3$, $\dot{y} = -xy^2 + 2x^5$.
  \end{enumerate}
\end{exercise}

\begin{solution}
  (1) 取 $V(x,y) = \frac12 (x^2+y^2)$, 则
  \[\frac{\diff V}{\diff t} = -x^2 y^2 - x^4y^2 \leq 0.\]
  因此零解是稳定的.

  (2) 取 $V(x,y) = \frac{1}{4}(x^4+y^4)$, 则
  \[\frac{\diff V}{\diff t} = -(x^8+y^8) \leq 0.\]
  因此零解是稳定的.

  (3) 取 $V(x,y) = \frac14 x^4 + \frac12 y^2$, 则在 $(0,0)$ 的小邻域中有
  \[\frac{\diff V}{\diff t} = (x^4+y^4) \bigl(2(x+y)^2 - 1\bigr) \leq 0.\]
  因此零解是稳定的.

  (4) 取 $V(x,y) = xy$, 则
  \[\frac{\diff V}{\diff t} = x^2y^2 + y^4 + 2x^6.\]
  因此 $\frac{\diff V}{\diff t}$ 是正定的, 但 $V$ 本身不是半负定的,
  由 \cite[定理 3.3]{Ma-Zhou-Li} 知零解是不稳定的.
\end{solution}


\section{平面上的动力系统, 奇点与极限环}

\begin{exercise}
  设线性系统 (8.22) 以 $(0,0)$ 为高阶奇点, 试作出其相图.
\end{exercise}


\begin{exercise}
  判断下列方程的奇点 $(0,0)$ 的类型, 并作出该奇点附近的相图:
  \begin{enumerate}[(1)]
    \item $\dot{x} = 4y-x$, $\dot{y} = -9x+y$;
    \item $\dot{x} = 2x+y+xy^2$, $\dot{y} = x+2y+x^2+y^2$;
    \item $\dot{x} = 2x+4y+\sin y$, $\dot{y} = x+y+\e^y-1$;
    \item $\dot{x} = x+2y$, $\dot{y} = 5y-2x+x^3$;
    \item $\dot{x} = x(1-y)$, $\dot{y} = y(1-x)$.
  \end{enumerate}
\end{exercise}


\begin{exercise}
  设函数 $P(x,y)$ 和 $Q(x,y)$ 在单连通区域 $D$ 内连续可微, 且
  \[\frac{\partial P}{\partial x} + \frac{\partial Q}{\partial y} \neq 0,
    \qquad (x,y)\in D.\]
  试证系统
  \[\dot{x} = P(x,y),\quad \dot{y} = Q(x,y)\]
  在 $D$ 内不存在闭轨线.
\end{exercise}