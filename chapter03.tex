\chapter{存在和唯一性定理}



\section{皮卡存在和唯一性定理}



\begin{exercise}
  利用右端函数的性质讨论下列微分方程满足初值条件 $y(0)=0$ 的解的唯一性问题:
  \begin{enumerate}[(1)]
  \item $\frac{\diff y}{\diff x}=|y|^{\alpha}(\alpha>0)$;
  \item $\frac{\diff y}{\diff x}=\begin{cases}0,&\mbox{当}y=0,\\y\ln|y|,&\mbox{当}y\neq0.\end{cases}$
  \end{enumerate}
\end{exercise}

\begin{solve}
(1)显然$y=0$是满足初值条件的解,当$0<\alpha<1$时,由下式
\[\int_0^y\frac{\diff y}{|y|^{\alpha}}=x\]
确定的函数$\varPhi(x,y)$是满足初值条件的解,且当$x\neq0$时,$y\neq0$,故此时解不唯一.

当$\alpha\geq1$时,满足初值条件的解是唯一的,用反证法,假设还存在另一解$y=y(x),y(0)=0$,
则必存在$x_0$和$\epsilon$使得$y(x_0)=0$,且当$x_0<x<x_0+\epsilon$时,$y(x)\neq0$,故
\[\frac{1}{|y(x)|^{\alpha}}\frac{\diff y(x)}{\diff x}=1,x_0<x<x_0+\epsilon\]
因此\[\int_0^{y(x)}\frac{\diff y}{|y|^{\alpha}}=\int_{x_0}^x\frac{1}{|y(x)|^{\alpha}}\frac{\diff y(x)}{\diff x}\diff x=x-x_0<\infty\]
矛盾,故满足初值条件的解是唯一的.综上,当$0<\alpha<1$时,解不唯一;当$\alpha\geq1$时,解唯一.

(2)显然$y=0$是满足初值条件的解,下面用反证法证明满足初值条件的解是唯一的,
假设还存在另一解$y=y(x),y(0)=0$,则必存在$x_0$和$\epsilon$使得$y(x_0)=0$,且当$x_0<x<x_0+\epsilon$时,$y(x)\neq0$,故
\[\frac{1}{y(x)\ln|y(x)|}\frac{\diff y(x)}{\diff x}=1,x_0<x<x_0+\epsilon\]
因此\[\int_0^{y(x)}\frac{\diff y}{y\ln|y|}=\int_{x_0}^x\frac{1}{y(x)\ln|y(x)|}\frac{\diff y(x)}{\diff x}\diff x=x-x_0<\infty\]
矛盾,故满足初值条件的解是唯一的.
\end{solve}


2.试求初值问题:\[\frac{\diff y}{\diff x}=x+y+1,y(0)=0\]
的皮卡序列,并由此取极限求解.

\begin{solve}利用皮卡序列迭代公式
\[y_{n+1}(x)=y_0+\int_{x_0}^xf(x,y_n(x))\diff x\]
知\[y_1(x)=\int_0^x(x+1)\diff x=\frac{1}{2}x^2+x\]
\[y_2(x)=\int_0^x\left(\frac{1}{2}x^2+2x+1\right)\diff x=\frac{x^3}{3!}+x^2+x\]
\[y_3(x)=\int_0^x\left(\frac{x^3}{3!}+x^2+2x+1\right)\diff x=\frac{x^4}{4!}+\frac{x^3}{3}+x^2+x\]
\[y_4(x)=\int_0^x\left(\frac{x^4}{4!}+\frac{x^3}{3}+x^2+2x+1\right)\diff x=\frac{x^5}{5!}+\frac{x^4}{12}+\frac{x^3}{3}+x^2+x\]
观察规律并用归纳法可证得
\[y_n(x)=\frac{x^{n+1}}{(n+1)!}+\frac{2x^n}{n!}+\frac{2x^{n-1}}{(n-1)!}+\cdots+\frac{2x^2}{2}+x\]
故\[\lim_{n\to\infty}y_n(x)=2\e^x-x-2\qedhere\]
\end{solve}


*3.设连续函数$f(x,y)$对$y$是递减的,则初值问题$(E)$在右侧(即$x\geq x_0$)的解是唯一的.(试问:在左侧(即$x\leq x_0$)的解是否唯一?能举一个反例吗?)
\begin{proof}
(反证法)假设初值问题有两个解$y_1(x)$和$y_2(x)$,且存在$x_1>x_0$使得$y_1(x_1)\neq y_2(x_1)$,不妨设$y_1(x_1)>y_2(x_1)$,记
\[\bar{x}=\sup\{x_0\leq x<x_1:y_1(x)=y_2(x)\}\]
显然有$x_0\leq\bar{x}<x_1$,令$r(x)=y_1(x)-y_2(x)$,则$r(\bar{x})=0$且当$\bar{x}<x<x_1$时$r(x)>0$,又
\[r'(x)=y_1'(x)-y_2'(x)=f(x,y_1(x))-f(x,y_2(x))<0,\bar{x}<x<x_1\]
所以$r(x)\leq0(\bar{x}<x<x_1)$,矛盾,故假设不成立,即证初值问题在右侧的解是唯一的,左侧的解不唯一,例如方程$\frac{\diff y}{\diff x}=-\frac{3}{2}y^{\frac{1}{3}}$,其解为$y^2=(-x+C)^3$以及特解$y=0$,显然过$x$轴上每一点的左侧解不是唯一的.\\
注:若$f(x,y)$对$y$是递增的,同理可以证明初值问题$(E)$在左侧的解是唯一的.
\end{proof}


\section{佩亚诺存在定理}


1.利用Ascoli引理证明: 若一函数序列在有限区间$I$上是一致有界和等度连续的, 则在$I$上它至少有一个一致收敛的子序列.

\begin{proof}
若$I=[a,b]$是有界闭区间, 则即为Ascoli定理, 下面假设函数序列$\{f_n(x)\}$在有界开区间$(a,b)$上是一致有界和等度连续的, 即假设:
\begin{enumerate}[(i)]
\item (一致有界)存在正常数$M>0$,使得
\[|f_n(x)|\leq M,\forall x\in(a,b),\forall n=1,2,\cdots\]
\item (等度连续)$\forall\epsilon>0,\exists\delta=\delta(\epsilon)>0,$使得$\forall x_1,x_2\in(a,b),|x_1-x_2|<\delta$,有
\[|f_n(x_1)-f_n(x_2)|<\epsilon,\forall n=1,2,\cdots\]
\end{enumerate}
由Cauchy收敛原理:
\[\lim_{x\to a+}f_n(x)存在\Leftrightarrow\forall\epsilon>0,\exists\delta>0,\forall x_1,x_2\in\{x|0<x-a<\delta\},|f_n(x_1)-f_n(x_2)|<\epsilon\]
结合$\{f_n(x)\}$等度连续知对于每一个$n$,单侧极限$\displaystyle\lim_{x\to a+}f_n(x)$存在, 
同理单侧极限$\displaystyle\lim_{x\to b-}f_n(x)$存在.定义函数序列
\[F_n(x)=\begin{cases}
\lim\limits_{x\to a+}f_n(x),&x=a,\\
f_n(x),&x\in (a,b),\\
\lim\limits_{x\to b-}f_n(x),&x=b.
\end{cases}n=1,2,\cdots\]
则$\{F_n(x)\}$在闭区间$[a,b]$上一致有界且等度连续,因此其在区间$[a,b]$上至少有一个一致收敛的子序列,
由此可知$\{f_n(x)\}$在$(a,b)$上至少有一个一致收敛的子序列.
\end{proof}


2.试举例说明,当$I$是无限区间时上面的结论不成立.

\begin{solve}
取定义在$[0,+\infty)$上的函数序列
\[f_n(x)=\begin{cases}
\frac{x}{n},&0\leq x\leq n,\\
1,&x>n.
\end{cases}\]
容易验证$\{f_n(x)\}$是一致有界且等度连续的,下面证明$\{f_n(x)\}$没有一致收敛的子序列$\{f_{n_k}(x)\}$,首先若$\{f_{n_k}(x)\}$一致收敛,则必收敛到0,但是
\[\lim_{k\to\infty}d(f_{n_k},0)=\lim_{k\to\infty}\sup_{x\in[0,+\infty)}|f_{n_k}(x)-0|=1\neq0\]
矛盾,故$\{f_n(x)\}$没有一致收敛的子序列.
\end{solve}


3.我们知道: 皮卡序列满足 Ascoli 引理的条件.试问: 能用皮卡序列来证明佩亚诺的存在定理吗?说明理由.

\begin{solve}
不能.因为在定义皮卡序列的积分式中:
\[y_{n+1}(x)=y_0+\int_{x_0}^xf(x,y_n(x))\diff x\]
$y_{n+1}(x)$通过$y_n(x)$表示出来,一旦限制在子序列上,这种表示法就失效了.
\end{solve}


4.对于与初值问题$(E)$等价的积分方程
\[y(x)=y_0+\int_{x_0}^xf(x,y(x))\diff x\]
在区间$I=[x_0,x_0+h]$上(其中正数$h$的意义同定理3.3)构造序列$y_n(x)$如下:任给正整数$n$,令$x_k=x_0+kd_n$,其中$d_n=h/n,k=0,1,\cdots,n$.则分点
\[x_0,x_1,x_2,\cdots,x_n(=x_0+h)\]
把区间$I$分成$n$等份.从$[x_0,x_1]$到$[x_1,x_2]$,再从$[x_1,x_2]$到$[x_2,x_3],\cdots$,最后从$[x_{n-2},x_{n-1}]$到$[x_{n-1},x_0+h]$用递推法定义下面的函数:
\[y_n(x)=\begin{cases}
y_0,&x\in[x_0,x_1];\\
y_0+\int_{x_0}^{x-d_n}f(s,y_n(s))\diff s,&x\in[x_1,x_0+h].
\end{cases}\]
称序列$y_1(x),y_2(x),\cdots,y_n(x),\cdots(x\in I)$为Tonelli序列,试用Tonelli序列和Ascoli引理证明佩亚诺存在定理.

\begin{proof}
由定义式可知$y_n(x)$是如下递推得到的:

当$x\in[x_0,x_1]$时
\[y_n(x)=y_0\]

当$x\in[x_1,x_2]$时
\[y_n(x)=y_0+\int_{x_0}^{x-d_n}f(s,y_n(s))\diff s=y_0+\int_{x_0}^{x-d_n}f(s,y_0)\diff s\]

当$x\in[x_2,x_3]$时
\[y_n(x)=y_0+\int_{x_0}^{x-d_n}f(s,y_n(s))\diff s=y_0+\int_{x_0}^{x_1}f(s,y_0)\diff s+\int_{x_1}^{x-d_n}f(s,y_n(s))\diff s\]
(其中上式中最右侧积分式里的$y_n(s)$为当$x\in[x_1,x_2]$时已经得到的$y_n(x)$)

这样不断地递推下去,即可得到$y_n(x)$在区间$[x_0,x_0+h]$上面的表达式,
容易验证$\{y_n(x)\}$在区间$[x_0,x_0+h]$上面连续,且由$y_n(x)-y_0=0,x\in[x_0,x_1]$和
\[|y_n(x)-y_0|=\left|\int_{x_0}^{x-d_n}f(s,y_n(s))\diff s\right|\leq M(x-d_n-x_0)(x_1\leq x\leq x_0+h)\]
知$\{(x,y_n(x))|x\in[x_0,x_0+h]\}\subset R$.

对于$\forall x,\tilde{x}\in[x_0,x_0+h]$,不妨设$x<\tilde{x}$,则$\max\{\tilde{x}-d_n,x_0\}\geq\max\{x-d_n,x_0\}$,由定义知
\[y_n(x)=y_0+\int_{x_0}^{\max\{x-d_n,x_0\}}f(s,y_n(s))\diff s\]
\[y_n(\tilde{x})=y_0+\int_{x_0}^{\max\{\tilde{x}-d_n,x_0\}}f(s,y_n(s))\diff s\]
故
\[\begin{split}|y_n(\tilde{x})-y_n(x)|&=\left|\int_{\max\{x-d_n,x_0\}}^{\max\{\tilde{x}-d_n,x_0\}}f(s,y_n(s))\diff s\right|\\
&\leq M(\max\{\tilde{x}-d_n,x_0\}-\max\{x-d_n,x_0\})\leq M(\tilde{x}-x)
\end{split}\]
因此$\{y_n(x)\}$在区间$[x_0,x_0+h]$上等度连续,由Ascoli定理知$\{y_n(x)\}$有一致收敛的子序列
$\{y_{n_k}(x)\}$且$y_{n_k}(x)\Rightarrow\phi(x)$,在下面定义式中
\[y_{n_k}(x)=y_0+\int_{x_0}^{\max\left\{x-d_{n_k},x_0\right\}}f(s,y_{n_k}(s))\diff s\]
取极限$k\to\infty$,注意到$\max\left\{x-d_{n_k},x_0\right\}\to x$,即得
\[\phi(x)=y_0+\int_{x_0}^xf(s,\phi(s))\diff s\]
故$\phi(x)$是初值问题$(E)$的一个解.
\end{proof}


5.题目有问题,跳过此题.


\section{解的延伸}


1.利用定理3.5证明:线性微分方程
\[\frac{\diff y}{\diff x}=a(x)y+b(x)(x\in I)\]
的每一个解$y=y(x)$的(最大)存在区间为$I$,这里假设$a(x)$和$b(x)$在区间$I$上是连续的.

\begin{proof}显然
\[|a(x)y+b(x)|\leq|a(x)||y|+|b(x)|\]
令$A(x)=|a(x)|\geq0,B(x)=|b(x)|\geq0$,则$A(x)$和$B(x)$都是$I$上的连续函数,由定理3.5知每一个解的最大存在区间为$I$.
\end{proof}


2.讨论下列微分方程解的存在区间:
\begin{enumerate}[(1)]
\item $\displaystyle\frac{\diff y}{\diff x}=\frac{1}{x^2+y^2}$;
\item $\displaystyle\frac{\diff y}{\diff x}=y(y-1)$;
\item $\displaystyle\frac{\diff y}{\diff x}=y\sin(xy)$;
\item $\displaystyle\frac{\diff y}{\diff x}=1+y^2$.
\end{enumerate}

\begin{solve}
(1)因为$\frac{1}{x^2+y^2}$在区域$\mathbb{R}^2\backslash\{0\}$上连续,
故由解的延伸定理知任意积分曲线必延伸到$\mathbb{R}^2\backslash\{0\}$的边界.设$y=y(x),x\in J$为一个饱和解,则
\[\frac{\diff y(x)}{\diff x}=\frac{1}{x^2+y^2(x)}>0\]
故存在反函数$x=x(y)$,且$x(y)$满足
\[\frac{\diff x(y)}{\diff y}=x^2(y)+y^2\]
由教材例1知$x=x(y)$的存在区间有限,不妨记为$(c,d)$,则当$y\to c+$或$y\to d-$时,$x(y)\to\infty$,也就说明$y(x)$有界,故其存在区间为$(-\infty,+\infty)$或$(-\infty,0)$或$(0,+\infty)$.\\
注:对照教材例1可以证明一般结论:微分方程
\[\frac{\diff y}{\diff x}=af(x)+by^2(f(x)\uparrow>0,ab>0)\]
任一解的存在区间都是有界的.

(2)这是变量分离的方程,容易解得方程的通解为$\displaystyle y(x)=\frac{1}{1-C\e^x}(C\in\mathbb{R})$,另外有特解$y=0$.

当$C<0$时,$0<y(x)<1$且$y(x)$单调减,当$x\to-\infty$时,$y(x)\to1$,当$x\to+\infty$时,$y(x)\to0$;

当$C=0$时,$y(x)=1$;

当$C>0$时,分母有零点$x_c=-\ln C$,当$x\in(-\infty,x_c)$时,$y(x)>0$单调增,$x\to-\infty$时,
$y(x)\to1$,$x\to x_c$时,$y(x)\to+\infty$;当$x\in(x_c,+\infty)$时,$y(x)<0$单调增,$x\to x_c$时,
$y(x)\to-\infty$,$x\to+\infty$时,$y(x)\to0$.

综上,解的存在区间为$(-\infty,+\infty)$或$(-\infty,-\ln C)$或$(-\ln C,+\infty)$.

(3)因为$y\sin(xy)$在$\mathbb{R}^2$上连续,且$|y\sin(xy)|\leq|y|$,故由定理3.5知解的存在区间为$(-\infty,+\infty)$.

(4)原方程的解为$x=\arctan y+C$,故解的存在区间为$(C-\frac{\pi}{2},C+\frac{\pi}{2})$.
\end{solve}


3.考虑对称形式的微分方程$x\diff x+y\diff y=0$,
它的定义域为$G=\{(x,y):x^2+y^2>0\}$.则单位圆$(x^2+y^2=1)$是一条积分曲线,
它在区域$G$的内部;它并没有延伸到$G$的边界,这一点是否与解的延伸定理相矛盾?为什么?

\begin{solve}
$\frac{\diff y}{\diff x}=-\frac{x}{y}$,因为$-\frac{x}{y}$在区域$G$上不连续,故不能运用延伸定理.
\end{solve}


4.设初值问题
\[(E):\frac{\diff y}{\diff x}=(y^2-2y-3)\e^{(x+y)^2},y(x_0)=y_0\]
的解的最大存在区间为:$a<x<b$,其中$(x_0,y_0)$是平面上任一点.则$a=-\infty$和$b=\infty$中至少有一个成立.

\begin{proof}
因为$f(x,y)=(y^2-2y-3)\e^{(x+y)^2}$在$\mathbb{R}^2$上连续,
且对$y$有连续的偏导数,故经过任一点的积分曲线唯一,显然$y=3$和$y=-1$是两个特解,其他积分曲线不与其二者相交,故:

(1)$y_0<-1$时,$\frac{\diff y}{\diff x}>0$,故解$y=y(x)$严格单调增,但不能与$y=-1$相交,故必有$b=+\infty$.

(2)$-1<y_0<3$时,$\frac{\diff y}{\diff x}<0$,故解$y=y(x)$严格单调减,但不能与$y=-1$和$y=3$相交,故必有$a=-\infty,b=+\infty$.

(3)$y_0>3$时,$\frac{\diff y}{\diff x}>0$,故解$y=y(x)$严格单调增,但不能与$y=3$相交,故必有$a=-\infty$.

(4)$y_0=-1$或$y_0=3$时,显然$a=-\infty,b=+\infty$.
\end{proof}


5.设初值问题
\[(E):\frac{\diff y}{\diff x}=(x^2-y^2)f(x,y),y(x_0)=y_0\]
其中函数$f(x,y)$在全平面连续且满足$yf(x,y)>0,\mbox{当}y\neq0$.则对于任意的$(x_0,y_0)$,
当$x_0<0$和$|y_0|$适当小时$(E)$的解可延拓到$-\infty<x<+\infty$.

\begin{proof}
显然$y=\pm x$是线素场的水平等斜线,由$f(x,y)$连续以及$yf(x,y)>0,\mbox{当}y\neq0$可知$f(x,0)=0$,
故$y=0$为方程的特解.当$x_0<0$且$|y_0|<|x_0|$时(以$y_0>0$为例),
解$y=y(x)$在区域$\{(x,y)|0\leq y<-x,x<0\}$单调增加,故可向左延伸至$-\infty$,
向右穿过$y=-x$后单调减,必与$y=x$相交,穿过直线$y=x$后单调增加且不能再次穿过$y=x$,故可向右延拓至$+\infty$.
\end{proof}


\section{比较定理及其应用}


1.设初值问题$(E)$,矩形区域$R$,和正数$h$的意义同定理3.1.
试证在$(E)$的最小解$y=W(x)$和最大解$y=Z(x)$之间充满了$(E)$的其它解,即任取一点$(x_1,y_1)$,其中
\[|x_1-x_0|\leq h,W(x_1)\leq y_1\leq Z(x_1),\]
则$(E)$在$|x-x_0|\leq h$上至少有一个解$y=u(x)$满足:$u(x_1)=y_1$.

\begin{proof}
为证明简单以$x_1\in(x_0,x_0+h]$为例,由解的延伸定理知初值问题:
\[(E_1):\frac{\diff y}{\diff x}=f(x,y),y(x_1)=y_1\]
的解$y=\phi(x)$必与$y=W(x)$或$y=Z(x)$相交,不妨设与$y=W(x)$相交于点$(\xi,W(\xi))$且两曲线在交点处相切,令
\[u(x)=\begin{cases}
W(x),&x_0-h\leq x\leq\xi\\
\phi(x),&\xi<x\leq x_0+h
\end{cases}\]
则$y=u(x)$为$(E)$的解且满足$u(x_1)=y_1$.
\end{proof}


2.证明(第二比较定理)设函数$f(x,y)$与$F(x,y)$都在平面区域$G$内连续且满足
\[f(x,y)\leq F(x,y),(x,y)\in G;\]
又设函数$y=\phi(x)$与$y=\varPhi(x)$在区间$a<x<b$上分别是初值问题\[(E_1):\frac{\diff y}{\diff x}=f(x,y),y(x_0)=y_0\]
与\[(E_2):\frac{\diff y}{\diff x}=F(x,y),y(x_0)=y_0\]
的解$[(x_0,y_0)\in G]$, 并且$y=\phi(x)$是$(E_1)$的右行最小解和左行最大解(或者: $y=\varPhi(x)$是$(E_2)$的右行最大解和左行最小解), 则有如下比较关系:
\[\phi(x)\leq\varPhi(x),x_0\leq x<b;\]
\[\phi(x)\geq\varPhi(x),a<x\leq x_0.\]

\begin{proof}不妨设$\varPhi(x)$是最大右行解和最小左行解, 为证当$x_0\leq x<b$时, 
$\phi(x)\leq\varPhi(x)$, 只需证明$\forall c\in(x_0,b)$, 有
\[\phi(x)\leq\varPhi(x),x\in[x_0,c]\]
考虑初值问题
\[(E_n^*):\frac{\diff y}{\diff x}=F(x,y)+\frac{1}{n},y(x_0)=y_0,n=1,2,\cdots\]
由Peano存在定理和解的延伸定理知$(E_n^*)$在$[x_0,c]$上有解, 取其中一解记为$\varPhi_n(x)$, 
由第一比较定理得$\phi(x)<\varPhi_n(x)<\varPhi_{n-1}(x),x_0\leq x\leq c$, 即
\[\varPhi_1(x)>\varPhi_2(x)>\cdots>\varPhi_n(x)>\cdots>\phi(x),x_0\leq x\leq c\]
显然$\{\varPhi_n(x)\}$在$[x_0,c]$上一致有界, 且由
\[|\varPhi_n(x_1)-\varPhi_n(x_2)|\leq\left|\int_{x_1}^{x_2}\left(F(x,\varPhi_n(x))+\frac{1}{n}\right)\diff x\right|\leq(M+1)|x_1-x_2|\]
知$\{\varPhi_n(x)\}$等度连续, 故其有一致收敛的子列, 不妨设其一致收敛:
\[\lim_{n\to\infty}\varPhi_n(x)=\varPhi_*(x)\]
在下式
\[\varPhi_n(x)=y_0+\int_{x_0}^x\left(F(x,\varPhi_n(x))+\frac{1}{n}\right)\diff x\]
中令$n\to\infty$, 得
\[\varPhi_*(x)=y_0+\int_{x_0}^xF(x,\varPhi_*(x))\diff x\]
故$\varPhi_*(x)$是$(E_2)$的解, 因此
\[\phi(x)\leq\lim_{n\to\infty}\varPhi_n(x)=\varPhi_*(x)\leq\varPhi(x),x_0\leq x\leq c\]
同理可证$\phi(x)\geq\varPhi(x),a<x\leq x_0$.
\end{proof}


3.设初值问题
\[\frac{\diff y}{\diff x}=x^2+(y+1)^2,y(0)=0\]
的解在右侧的最大存在区间为$[0,\beta)$,试证:$\frac{\pi}{4}<\beta<1$.
\begin{proof}
首先初值问题的解存在且唯一(记为$\phi(x)$)并可延伸到包含坐标原点的任意区域的边界,下面分三步证明$\frac{\pi}{4}<\beta<1$.\\
(1)证明:$\frac{\pi}{4}\leq\beta\leq1$.

当$|x|\leq1$时,显然有
\[(y+1)^2\leq x^2+(y+1)^2\leq1+(y+1)^2\]
\end{proof}